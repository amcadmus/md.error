\documentclass[aps,pre,preprint,unsortedaddress]{revtex4}
% \documentclass[aps,jcp,groupedaddress,twocolumn,unsortedaddress]{revtex4}

\usepackage{amsmath}
\usepackage{amssymb}
\usepackage[dvips]{graphicx}
\usepackage{color}
\usepackage{tabularx}

\makeatletter
\makeatother

\newcommand{\recheck}[1]{{\color{red} #1}}
\renewcommand{\v}[1]{\textbf{\textit{#1}}}
\renewcommand{\d}[1]{\textsf{#1}}


\begin{document}

\title{The Error Estimate of Force Calculation in Heterogeneous and Correlated Molecular Systems}
\author{Han Wang}
\affiliation{LMAM and School of Mathematical
  Sciences, Peking University}
\author{Pingwen Zhang}
% \email{pzhang@pku.edu.cn}
\affiliation{LMAM and School of Mathematical
  Sciences, Peking University}

\begin{abstract}
\end{abstract}

\maketitle


\section{Error estimate}
The error of force as a function of position is 
\begin{align} \nonumber
  \langle\vert\Delta\v F(\v r)\vert^2\rangle
  &= \big\langle\sum_{j,k}\v F^c(\v r - \v r_j)\cdot\v F^c(\v r - \v r_k)\big\rangle \\ \nonumber
  &= \big\langle\sum_j\vert\v F^c(\v r - \v r_j)\vert^2\big\rangle +
  \big\langle\sum_{j\neq k}\v F^c(\v r - \v r_j)\cdot\v F^c(\v r - \v r_k)\big\rangle \\ \label{eqn:tmp1}
  &= \int_{\mathbb R^3}\vert\v F^c(\v r - \v r')\vert^2\rho(\v r')\,\d d\v r'
  + \int_{\mathbb R^3\times\mathbb R^3}\v F^c(\v r - \v r')\cdot\v F^c(\v r - \v r'')\rho(\v r', \v r'')\,\d d\v r'\d d\v r'',
\end{align}
where $\v F^c$ is the complementary force defined by
\begin{align}
  \v F^c(\v r) =
  \left\{
  \begin{array}{ll}
    0, & \vert\v r\vert\leq r_c; \\
    \v F(\v r), & \vert\v r\vert > r_c.
  \end{array}
  \right.
\end{align}
The densities $\rho(\v r')$ and $\rho(\v r', \v r'')$ are related
to the probability densities $p(\v r')$ and $p(\v r', \v r'')$ by
\begin{align}
  \rho(\v r') &= N\,p(\v r') \\
  \rho(\v r', \v r'') &= N(N-1)\,p(\v r', \v r'')
\end{align}
The densities are periodically extented to the whole space $\mathbb
R^3$.
% The problem now this how to discretize the integrals in
% \eqref{eqn:tmp1}. With the piecewise constant numerical integration
% formulus, we have:
Now, let
\begin{align} \nonumber
  \rho(\v r', \v r'')
  & = N (N-1) \,p(\v r', \v r'') \\\nonumber
  & = N (N-1) \big\{ p(\v r')p(\v r'') + [\,p(\v r', \v r'') -  p(\v r')p(\v r'')\,]\big\} \\\nonumber
  &= \frac{N-1}N \rho(\v r')\rho(\v r'') + C (\v r', \v r'')\\ \label{eqn:tmp5}
  &\approx \rho(\v r')\rho(\v r'') + C (\v r', \v r'')
\end{align}
the function
\begin{align}
C (\v r', \v r'') = N (N-1) [\,p(\v r', \v r'') -  p(\v r')p(\v r'')\,]
\end{align}
denotes the correlation between
position $\v r'$ and $\v r''$. Therefore, \eqref{eqn:tmp1} becomes
\begin{align} \nonumber
  \langle\vert\Delta\v F^c(\v r)\vert^2\rangle
  = &\,
  \int_{\mathbb R^3}\vert\v F^c(\v r - \v r')\vert^2\rho(\v r')\,\d d\v r' \,+ \\\nonumber
  &\,
  \bigg[\int_{\mathbb R^3}\v F^c(\v r - \v r')\rho(\v r')\,\d d\v r'\,\bigg]^2 + \\\nonumber
  &\,
  \int_{\mathbb R^3\times\mathbb R^3}\v F^c(\v r - \v r')\cdot\v F^c(\v r - \v r'')\,C(\v r', \v r'')\,\d d\v r'\d d\v r'' \\\label{eqn:tmp5}
  = &\,
  \mathcal E_{\textrm{homo}}(\v r) + \mathcal E_{\textrm{hetero}}(\v r) + \mathcal E_{\textrm{corr}}(\v r)
\end{align}
This means the force error is a combination of a homogeneous
contribution, a heterogeneous contribution and a correlational
contribution. The heterogeneous force can be applied as a correction
to the error:
\begin{align}\label{eqn:tmp7}
  \v F_{\textrm{hetero}} = \int_{\mathbb R^3}\v F^c(\v r - \v r')\rho(\v r')\,\d d\v r'
\end{align}


\subsection{The homogeneous error}

The homogeneous error in \eqref{eqn:tmp5} can be calculated by:
\begin{align}\nonumber
  \mathcal E_{\textrm{homo}}(\v r)
  & = \int_{\mathbb R^3}\vert\v F^c(\v r - \v r')\vert^2\rho(\v r')\,\d d\v r' \\ \nonumber
  & = \sum_{\v n} \int_V \vert\v F^c(\v r - (\v r'+\v n))\vert^2\rho(\v r')\,\d d\v r'
\end{align}
So the error term is periodic, considering the Fourier transform of
this term:
\begin{align}\nonumber
  \hat{\mathcal E}_{\textrm{homo}}(\v k)
  & =
  \int_V\mathcal E_{\textrm{homo}}(\v r)e^{-2\pi i\v k\cdot\v r}\,\d d\v r \\ \nonumber
  & =
  \int_V
  \sum_{\v n} \int_V \vert\v F^c(\v r - (\v r'+\v n))\vert^2\rho(\v r')\,\d d\v r'\,
  e^{-2\pi i\v k\cdot\v r}\,\d d\v r
  % \\  \nonumber
  % & =
  % \int_V
  % \sum_{\v n} \int_V \vert\v F^c(\v r - (\v r'+\v n))\vert^2\rho(\v r')\,\d d\v r'\,
  % e^{-2\pi i\v k\cdot\v r}\,\d d\v r 
\end{align}
This is a Fourier transform of a convolution, so by omitting the details, we have
\begin{align}
  \hat{\mathcal E}_{\textrm{homo}}(\v k) = \hat K_{\textrm{homo}}(\v k)\,\hat\rho\,(\v k),
\end{align}
where $\hat K_{\textrm{homo}}(\v k)$ is the Fourier transform of
convolution kernel of homogeneous error $K_{\textrm{homo}}(\v r) = \vert\v F^c(\v
r)\vert^2$, namly:
\begin{align}
  \hat K_{\textrm{homo}}(\v k)
  &=
  \int_{\mathbb R^3}K_{\textrm{homo}}(\v r)e^{-2\pi i\v k\cdot\v r}\,\d d\v r
\end{align}
$\hat\rho\,(\v k)$ is the Fourier transform of the density, namely,
\begin{align}
  \hat\rho(\v k) = \int_V\rho(\v r)e^{-2\pi i\v k\cdot\v r}\,\d d\v r
\end{align}
Therefore,
\begin{align}\label{eqn:tmp11}
  \mathcal E_{\textrm{homo}}(\v r)
  &=
  \frac1V\sum_{-\infty}^\infty \hat{\mathcal E}_{\textrm{homo}}(\v k)\,e^{2\pi i\v k\cdot\v r}
\end{align}

All the Fourier transforms should be accelerated by fast Fourier
transform (FFT). Consider the lattice points in real space and
reciprocal space:
\begin{align}
  \v r_{i_1, i_2, i_3} & =
  \frac{i_1}{K_1}\v a_1 + 
  \frac{i_2}{K_2}\v a_2 + 
  \frac{i_3}{K_3}\v a_3, \\
  \v k_{k_1, k_2, k_3} & =
  k_1\v a_1^\ast +
  k_2\v a_2^\ast +
  k_3\v a_3^\ast,
\end{align}
where $\v a_\alpha$ are box vectors, and $\v a_\alpha^\ast$ are
reciprocal box vectors defined by $\v a_\alpha\cdot\v a_\beta^\ast =
\delta_{\alpha\beta}$. By a piecewise constant approximation of
\eqref{eqn:tmp11}, we have
\begin{align}\nonumber
  \mathcal E_{\textrm{homo}}(\v r_{i_1,i_2,i_3})
  & \approx
  \frac1V\sum_{k_1=-\infty}^\infty\sum_{k_2=-\infty}^\infty\sum_{k_3=-\infty}^\infty
  \hat{\mathcal E}_{\textrm{homo}}(\v k_{k_1,k_2,k_3})
  % e^{2\pi i\v k_{k_1,k_2,k_3}\cdot\v r_{i_1,i_2,i_3}}
  \exp\bigg[
  2\pi i\bigg(
  \frac{k_1i_1}{K_1} + \frac{k_2i_2}{K_2} + \frac{k_3i_3}{K_3}
  \bigg)
  \bigg] \\
  & \approx
  \frac1V\sum_{k_1=0}^{K_1}\sum_{k_2=0}^{K_2}\sum_{k_3=0}^{K_3}
  \hat{K}_{\textrm{homo}}(\v k'_{k_1,k_2,k_3})\:
  \hat{\rho}(\v k_{k_1,k_2,k_3})
  % e^{2\pi i\v k_{k_1,k_2,k_3}\cdot\v r_{i_1,i_2,i_3}}
  \exp\bigg[
  2\pi i\bigg(
  \frac{k_1i_1}{K_1} + \frac{k_2i_2}{K_2} + \frac{k_3i_3}{K_3}
  \bigg)
  \bigg]   
\end{align}
The prime on $\v k_{k_1,k_2,k_3}$ means that we should use the
periodic image $k_\alpha - K_\alpha$ instead of $k_\alpha$ when $k_\alpha \geq
K_\alpha/2$. Also by piecewise constant approximation, The Fourier
transform of density is 
\begin{align}\nonumber
  \hat\rho(\v k_{k_1,k_2,k_3})
  &=
  \int_V\rho(\v r)e^{-2\pi i\v k_{k_1,k_2,k_3}\cdot\v r}\,\d d\v r \\
  & =
  \frac{V}{K_1K_2K_3}
  \sum_{i_1=0}^{K_1}\sum_{i_2=0}^{K_2}\sum_{i_2=0}^{K_2}
  \rho(\v r_{i_1,i_2,i_3})
  \exp\bigg[
  -2\pi i\bigg(
  \frac{k_1i_1}{K_1} + \frac{k_2i_2}{K_2} + \frac{k_3i_3}{K_3}
  \bigg)
  \bigg]
\end{align}
Since the in most cases the force is isotropoic, then the homogeneous
kernel is isotrpoic, so it is a function of the absolute value of $\v
r$: $K_{\textrm{homo}}(\v r) = K_{\textrm{homo}}(r)$. The Fourier
transform of this kind of kernel can be simplified as
\begin{align}
  \hat{K}_{\textrm{homo}}(\v k_{k_1,k_2,k_3})
  =
  \int_{\mathbb R^3}K_{\textrm{homo}}(r)e^{-2\pi i\v k\cdot\v r}\d d\v r
  =
  \int_{r_c}^\infty \frac{2 r}k\, K_{\textrm{homo}}(r) \sin(2\pi kr)\,\d dr
\end{align}
This is now calculated numerically. (\recheck{Probably there exists an
analytical solution.})


\subsection{The heterogeneous error}
From \eqref{eqn:tmp7},
% by denoting the force by the heterogeneous kernel $\v K_{\textrm{hetero}}$,
% \begin{align}\nonumber
%   \v F_{\textrm{hetero}}
%   &=
%   \int_{\mathbb R^3}\v F^c(\v r - \v r')\rho(\v r')\,\d d\v r'\\
%   &=
%   \int_{\mathbb R^3}\v K_{\textrm{hetero}}(\v r - \v r')\rho(\v r')\,\d d\v r'  
% \end{align}
all the same as the homogeneous error
\begin{align}
  \hat{\v F}_{\textrm{hetero}}(\v k) = \hat{\v F}^c(\v k)\,\hat\rho(\v k).
\end{align}
Since
\begin{align}
  \v F^c(\v r) = - \nabla U^c(\v r),
\end{align}
where $U^c(\v r)$ is the truncated potential. The Fourier transform of
heterogeneous kernel is
\begin{align}\nonumber
  -\hat{\v F}^c(\v k) 
  & =
  \int_{\Omega_c}\nabla U(\v r) e^{-2\pi i\v k\cdot\v r}\d d\v r \\ \label{eqn:tmp19}
  & =
  \int_{\partial\Omega_c}U(\v r) e^{-2\pi i\v k\cdot\v r} \v n\d dS
  + 2\pi i \v k\int_{\Omega_c} U^c(\v r) e^{-2\pi i\v k\cdot\v r}\,\d d\v r
\end{align}
For the first integral, set up a spherical constant system so that the
$z$-axis is along the vector $\v k$, then
\begin{align}
  \int_{\partial\Omega_c}U(\v r) e^{-2\pi i\v k\cdot\v r} \v n\,\d dS
  & =
  - \int_0^{2\pi}\int_0^{\pi}
  U(r_c) e^{-2\pi k r_c\cos\phi}
\left(
\begin{aligned}[m]
  &\sin\phi\cos\theta\\
  &\sin\phi\sin\theta\\
  &\cos\phi
\end{aligned}
\right)
r_c^2\sin\phi\:
\d d\phi\,
\d d\theta
\end{align}
By the symmetricity, the first two component of the integral are 0. So
the first integral of \eqref{eqn:tmp19} is along the direction of
$k$. The maganitude of this integral is:
\begin{align}\nonumber
  &- \int_0^{2\pi}\int_0^{\pi}
  U(r_c) e^{-2\pi k r_c\cos\phi}
  \cos\phi\,
  r_c^2\sin\phi\:
  \d d\phi\,
  \d d\theta\\\nonumber
  = &
  -2\pi r_c^2\,U(r_c)
  \int_0^\pi\sin\phi\cos\phi
  \big[\cos(2\pi kr\cos\phi) - i\sin(2\pi kr\cos\phi)\big]
  \,\d d\phi \\\nonumber
  = &
  -2\pi r_c^2\,U(r_c)
  \int_{-1}^1 x \big[\cos(2\pi krx) - i\sin(2\pi krx)\big]
  \,\d dx \\ \nonumber
  = &\;
  -2\pi i\, r_c^2\,U(r_c)
  \int_{-1}^1 -x\sin(2\pi krx)
  \,\d dx \\ 
  = &\;
  -2\pi i\, r_c^2\,U(r_c)
  \bigg[\,
  \frac{2\cos(2\pi kr)}{2\pi kr}
  -\frac{2\sin(2\pi kr)}{(2\pi kr)^2}
  \,\bigg]
\end{align}
Therefore, \eqref{eqn:tmp19} becomes
\begin{align}\nonumber
  -\hat{\v F}^c(\v k) 
  & =
  \int_{\partial\Omega_c}U(\v r) e^{-2\pi i\v k\cdot\v r} \v n\d dS
  + 2\pi i \v k\int_{\Omega_c} U^c(\v r) e^{-2\pi i\v k\cdot\v r}\,\d d\v r\\ \nonumber
  & = 
  -2\pi i\, r_c^2\,U(r_c)
  \bigg[\,
  \frac{2\cos(2\pi kr)}{2\pi kr}
  -\frac{2\sin(2\pi kr)}{(2\pi kr)^2}
  \,\bigg] \,\frac{\v k}k +
  2\pi i \v k
  \int_{r_c}^\infty \frac{2r}k\,U(r)\sin(2\pi kr)\d d\v r \\
  & =
  2\pi\frac{\v k}k\, i\,
  \bigg\{
  \int_{r_c}^\infty 2r\,U(r)\sin(2\pi kr)\d d\v r
  - r_c^2\,U(r_c)
  \bigg[\,
  \frac{2\cos(2\pi kr)}{2\pi kr}
  -\frac{2\sin(2\pi kr)}{(2\pi kr)^2}
  \,\bigg] 
  \bigg\}
\end{align}



\section{Energy and pressure corrections}
The energy correction and pressure correction of the system are:
\begin{align}
  E^c &= \bigg\langle \sum_{i\neq j} U^c(\v r_i - \v r_j) \bigg\rangle\\
  \v P^c &= \frac1{V} \bigg\langle \sum_{i\neq j}\,\frac12\, (\v r_i - \v r_j) \otimes \v F^c(\v r_i - \v r_j) \bigg\rangle
\end{align}
For simplicity, let us consider the general form the the corrections:
\begin{align}
  Q = \bigg\langle \sum_{i\neq j} K(\v r_i - \v r_j) \bigg\rangle
\end{align}
For energy correction, the kernel $K$ is $U^c$, while for the pressure
correction, the kernel is $\v r\otimes\v F^c$.
Let
\begin{align}
  \rho(\v r', \v r'') =
  \bigg\langle
  \sum_{i\neq j}\delta(\v r' - \v r_i)\delta(\v r'' - \v r_j)
  \bigg\rangle,
\end{align}
then, by using \eqref{eqn:tmp5}, we have 
\begin{align}\nonumber
  Q
  &= \bigg\langle \sum_{i\neq j} K(\v r_i - \v r_j) \bigg\rangle\\\nonumber
  & = \int K(\v r' - \v r'') \rho(\v r', \v r'') \d d\v r'\d d\v r'' \\\nonumber
  & = \int K(\v r' - \v r'')
  \big[
  \rho(\v r') \rho(\v r'') + C(\v r', \v r'')
  \big]
  \d d\v r'\d d\v r''
\end{align}
Assuming $C(\v r', \v r'')$ vanishes when $\vert\v r' - \v r''\vert \geq r_c$, then
\begin{align} \nonumber
  Q 
  & = \int K(\v r' - \v r'')
  \rho(\v r') \rho(\v r'') 
  \d d\v r'\d d\v r'' \\
  & = \int
  \Big[
  \int K(\v r' - \v r'') \rho(\v r'')\,\d d\v r''
  \Big]
  \rho(\v r')\,\d d\v r'
\end{align}

The Fourier transform of the convolution $\int K(\v r' - \v r'')
\rho(\v r'')\,\d d\v r''$ can be easily calculated by $\hat K(\v
k)\,\hat \rho(\v k)$.  The Fourier transform of the energy correction
kernel, which is almost the same as the homogeneous error kernel
$K_{\textrm{homo}}$, can be easily calculated. The Fourier transform
of the pressure correction kernel requires more efforts.  For the
dispersion interaction, the $\alpha\beta$ component of the pressure
correction kernel can be written as
\begin{align}
  \{\v r\otimes\v F^c\}_{\alpha\beta} = r_\alpha r_\beta \,G^c(r) \qquad \alpha, \beta = 1, 2, 3
\end{align}
To calculate this Fourier transoform, we consider the spherical
coordinate by letting
\begin{align} \nonumber
  r_1 &= r\sin\phi\cos\theta \\\nonumber
  r_2 &= r\sin\phi\sin\theta \\\nonumber
  r_3 &= r\cos\phi
\end{align}
and 
\begin{align} \nonumber
  k_1 &= k\sin\Phi\cos\Theta \\\nonumber
  k_2 &= k\sin\Phi\sin\Theta \\\nonumber
  k_3 &= k\cos\Phi
\end{align}

For $\alpha = 1,\ \beta = 1$,
\begin{align}\nonumber
  [\,r_1r_1\,G^c(r)\,]^{\wedge} 
  &= \int_{\mathbb R^3} r_1r_1\,G^c(r)\, e^{-2\pi i\v k\cdot\v r}\d d\v r \\\nonumber
  &= \int_{r_c}^\infty \d dr \int_{0}^{2\pi} \d d\theta \int_0^\pi\d d\phi\:
  r^2\sin\phi \,r^2 \sin^2\phi\cos^2\theta\,G(r)\,
  e^{-2\pi k r[\sin\phi\sin\Phi\cos(\theta - \Theta) + \cos\phi\cos\Phi]} \\
  &= \int_{r_c}^\infty \d dr \int_{0}^{2\pi} \d d\theta \int_0^\pi\d d\phi\:
  r^4\sin^3\phi\,G(r)\,e^{-2\pi i kr\cos\phi\cos\Phi}
  \cos^2\theta\,e^{-2\pi i kr\sin\phi\sin\Phi\cos(\theta- \Theta)}
\end{align}
Let $c = 2\pi kr\sin\phi\sin\Phi$, consider the integral of $\theta$
\begin{align}\nonumber
  \int_0^{2\pi} \cos^2\theta\,e^{-ci\cos(\theta - \Theta)}\d d\theta
  & =
  \int_{0-\Theta}^{2\pi-\Theta} \cos^2(t + \Theta) \, e^{-ci\cos t}\d dt \\\nonumber
  & = 
  \int_0^{2\pi}\cos^2(t + \Theta)\, e^{-ci\cos t} \d dt \\\nonumber
  & = 
  \int_0^{2\pi}\frac{1 + \cos[2(t + \Theta)]}2 \, e^{-ci\cos t} \d dt \\\nonumber  
  & = 
  \int_0^{2\pi}
  \Big(
  \frac12 + \frac12\cos 2t\cos 2\Theta - \frac12\sin 2t\sin 2\Theta
  \Big)
  \, e^{-ci\cos t} \d d t
\end{align}
Using
\begin{align}\label{eqn:tmp31}
  \int_0^{2\pi} \frac12 \,e^{-ci\cos t}\d dt &= \pi J_0(-c) \\\label{eqn:tmp32}
  \int_0^{2\pi} \frac12\cos 2t \,e^{-ci\cos t}\d dt &= -\pi J_2(-c) \\\label{eqn:tmp33}
  \int_0^{2\pi} \frac12\sin 2t \,e^{-ci\cos t}\d dt &= 0 
\end{align}
where $J_\nu(c)$ is the Bessel function, the integral definition of which is
\begin{align}
  J_\nu(x) =
  \frac
  {\big({\frac x2}\big)^\nu}
  {\Gamma(\nu + \frac12)\sqrt\pi}
  \int_{-1}^1e^{ixs}(1-s^2)^{\nu-\frac12}\d ds
\end{align}
So
\begin{align}
  J_\nu(-x) = (-1)^\nu J_\nu(x)
\end{align}

Consider the integral of \eqref{eqn:tmp31} over $\phi$, letting $b = -2\pi kr$,
\begin{align}\label{eqn:tmp34}
  \int_0^\pi
  \sin^3\phi\, e^{bi\cos\phi\cos\Phi}\,\pi J_0(b\sin\phi\sin\Phi)\,
  \d d\phi
\end{align}
We need the conclusion:
\begin{align}
  \int_0^\pi
  e^{ib\cos\phi\cos\Phi}
  J_{\nu-\frac12}(b\sin\phi\sin\Phi)\,
  C_r^\nu(\cos\phi)\,
  \sin^{\nu + \frac12}\phi
  \,\d d\phi
  =
  \Big(\frac{2\pi}{b}\Big)^{\frac12}
  i^r
  \sin^{\nu - \frac12}\Phi\,
  C_r^\nu(\cos\Phi)\,
  J_{\nu + r}(b)
\end{align}
where $C_r^\nu(x)$ is the Gegenbauer function, the first several
Gegenbauer functions and the interative definition are
\begin{align}
  C_0^\alpha (x) &= 1 \\
  C_1^\alpha(x) &= 2\alpha x\\
  C_2^\alpha(x) &= 2\alpha(\alpha + 1) x^2 - \alpha\\
  C_n^\alpha(x) & = \frac{1}{n}\,\Big[
  2x(n+\alpha-1)C_{n-1}^\alpha(x) - (n+2\alpha-2)C_{n-2}^\alpha(x)
  \Big].
\end{align}
Note $C_2^{1/2} = -1/2 + 3/2\,x^2$, we have
\begin{align}
  \sin^3\phi = \sin\phi(1-\cos^2\phi)=
  \sin\phi\,\frac23\,\Big[
  C_0^{\frac12}(\cos\phi) - C_2^{\frac12}(\cos\phi)
  \Big]
\end{align}
\eqref{eqn:tmp34} is equal to
\begin{align}\nonumber
  &\int_0^\pi
  \frac23\pi\sin\phi\,
  \Big[
  C_0^{\frac12}(\cos\phi) - C_2^{\frac12}(\cos\phi)
  \Big]
  \, e^{bi\cos\phi\cos\Phi}\, J_0(b\sin\phi\sin\Phi)\,
  \d d\phi   \\ \nonumber
  = \,&
  \frac23\, \pi\,
  \Big( \frac{2\pi}b \Big )^{\frac12}
  \Big[\,
  J_{\frac12}(b) - \frac12\,J_{\frac52}(b) +
  \frac32\,\cos^2\Phi\, J_{\frac52}(b)
  \,\Big] \\ \nonumber
  = \,&
  \frac23\, \pi\,
  \sqrt{ \frac1{kr} }(-i)
  \Big[\,
  iJ_{\frac12}(2\pi kr) -  \frac12\,i\,J_{\frac52}(2\pi kr) +
  \frac32\,i\,\cos^2\Phi\, J_{\frac52}(2\pi kr)
  \,\Big] \\
  = \,&
  \pi\,
  \sqrt{ \frac1{kr} }\,
  \Big[\,
  \frac23\,J_{\frac12}(2\pi kr) -  \frac13\,J_{\frac52}(2\pi kr) +
  \cos^2\Phi\, J_{\frac52}(2\pi kr)
  \,\Big] 
\end{align}

Consider the integral of \eqref{eqn:tmp32} is
\begin{align}\nonumber
  &-\int_0^\pi
  \sin^3\phi\, e^{bi\cos\phi\cos\Phi}\,\pi J_2(b\sin\phi\sin\Phi)\,
  \d d\phi \\ \nonumber
  = &\, 
  -\pi \Big( \frac{2\pi}b \Big )^{\frac12}\,
  \sin^2\Phi\,
  J_{\frac52}(b) \\
  = &\, 
  -\pi \sqrt{ \frac1{kr} }\,
  \sin^2\Phi\,
  J_{\frac52}(2\pi kr) 
\end{align}

Therefore,
\begin{align}\nonumber
  &[\,r_1r_1\,G^c(r)\,]^{\wedge} \\
  =\,&
  \int_{r_c}^\infty \d dr\, r^4 G(r)\,
  \pi \sqrt{ \frac1{kr} }\,
  \Big[\,
  \frac23\,J_{\frac12}(2\pi kr) -
  \big(\,
  \frac13 -
  \cos^2\Phi +
  \sin^2\Phi\cos2\Theta
  \big)\, J_{\frac52}(2\pi kr)
  \,\Big] 
\end{align}
Similarly, we have
\begin{align}\nonumber
  &[\,r_2 r_2\,G^c(r)\,]^{\wedge} \\
  =\,&
  \int_{r_c}^\infty \d dr\, r^4 G(r)\,
  \pi \sqrt{ \frac1{kr} }\,
  \Big[\,
  \frac23\,J_{\frac12}(2\pi kr) -
  \big(\,
  \frac13 -
  \cos^2\Phi -
  \sin^2\Phi\cos2\Theta
  \big)\, J_{\frac52}(2\pi kr)
  \,\Big] 
\end{align}

For $\alpha = 3,\ \beta = 3$,
\begin{align}\nonumber
  [\,r_3 r_3\,G^c(r)\,]^{\wedge} 
  &= \int_{\mathbb R^3} r_3 r_3\,G^c(r)\, e^{-2\pi i\v k\cdot\v r}\d d\v r \\\nonumber
  &= \int_{r_c}^\infty \d dr \int_{0}^{2\pi} \d d\theta \int_0^\pi\d d\phi\:
  r^2\sin\phi \,r^2 \cos^2\phi\,G(r)\,
  e^{-2\pi k r[\sin\phi\sin\Phi\cos(\theta - \Theta) + \cos\phi\cos\Phi]} \\
  &= \int_{r_c}^\infty \d dr \int_{0}^{2\pi} \d d\theta \int_0^\pi\d d\phi\:
  r^4\sin\phi\cos^2\phi\,G(r)\,e^{-2\pi i kr\cos\phi\cos\Phi}
  \,e^{-2\pi i kr\sin\phi\sin\Phi\cos(\theta- \Theta)}
\end{align}
Since
\begin{align}\nonumber
  \int_0^{2\pi} e^{-c i\cos(\theta - \Theta)} \d d\theta
  &=
  \int_0^{2\pi} e^{-c i\cos t} \d dt \\\nonumber
  &=
  2\pi J_0(-c)\\
  & =
  2\pi J_0(-2\pi kr\sin\phi\sin\Phi)
\end{align}
And
\begin{align}\nonumber
  &
  \int_0^\pi 2\pi \,\sin\phi\cos^2\phi\,
  J_0(b\sin\phi\sin\Phi) e^{b\cos\phi\cos\Phi} \,\d d\phi \\\nonumber
  = \,&
  \int_0^\pi 2\pi \,\sin\phi \:\frac23\,
  \Big[\,
  \frac12 C_0^{\frac12} (\cos\phi) + C_2^{\frac12}(\cos\phi)
  \,\Big]
  J_0(b\sin\phi\sin\Phi) e^{b\cos\phi\cos\Phi} \,\d d\phi \\ \nonumber
  = \,&
  2\pi \Big(\frac{2\pi}c \Big)^{\frac12} \:\frac23\:
  \Big[\,
  \frac12 C_0^{\frac12}(\cos\Phi) J_{\frac12}(b) -
  C_2^{\frac12} (\cos\Phi) J_{\frac52}(b)
  \,\Big] \\
  = \,&
  \frac43 \pi \,\sqrt{\frac1{kr}}\:
  \Big[\,
  \frac12 J_{\frac12}(2\pi kr) +
  \big(\,
  \frac12 - \frac32 \cos^2\Phi
  \big)
  J_{\frac52}(2\pi kr)
  \,\Big]
\end{align}
Therefore,
\begin{align}
  [\,r_3 r_3\,G^c(r)\,]^{\wedge} 
  =
  \int_{r_c}^\infty \d dr\: \frac43\,\pi\,r^4 G(r)\,
  \sqrt{ \frac1{kr} }\,
  \Big[\,
  \frac12 J_{\frac12}(2\pi kr) +
  \big(\,
  \frac12 - \frac32 \cos^2\Phi
  \big)
  J_{\frac52}(2\pi kr)
  \,\Big]
\end{align}

For $\alpha = 1,\ \beta = 2$,
\begin{align}\nonumber
  [\,r_1 r_2\,G^c(r)\,]^{\wedge} 
  &= \int_{\mathbb R^3} r_1 r_2\,G^c(r)\, e^{-2\pi i\v k\cdot\v r}\d d\v r \\\nonumber
  &= \int_{r_c}^\infty \d dr \int_{0}^{2\pi} \d d\theta \int_0^\pi\d d\phi\:
  r^2\sin\phi \,r^2 \sin^2\phi \sin\theta\cos\theta \,G(r)\,
  e^{-2\pi k r[\sin\phi\sin\Phi\cos(\theta - \Theta) + \cos\phi\cos\Phi]} \\
  &= \int_{r_c}^\infty \d dr \int_{0}^{2\pi} \d d\theta \int_0^\pi\d d\phi\:
  r^4\sin^3\phi\,G(r)\,e^{-2\pi i kr\cos\phi\cos\Phi}
  \sin\theta\cos\theta\,e^{-2\pi i kr\sin\phi\sin\Phi\cos(\theta- \Theta)}
\end{align}
Since
\begin{align}\nonumber
  \int_0^{2\pi}\sin\theta\cos\theta e^{-c i\cos(\theta - \Theta)} \d d\theta
  & =
  \int_0^{2\pi}\sin(t+\Theta)\cos(t+\Theta)
  e^{-c i \cos t} \d dt \\\nonumber
  &=
  \int_0^{2\pi}
  \Big[\,
  \frac12 \sin 2t\cos 2\Theta + \frac12\cos 2t\sin 2\Theta
  \,\Big]
  e^{-c i\cos t} \d dt \\\nonumber
  &=
  -\pi \sin 2\Theta J_2(-2\pi kr\sin\phi\sin\Phi)
\end{align}
And
\begin{align}\nonumber
  &
  \int_0^\pi -\pi \sin 2\Theta\,
  \sin^3\phi\,
  J_2(b\sin\phi\sin\Phi) e^{b\cos\phi\cos\Phi} \,\d d\phi \\\nonumber
  = \,&
  -\pi\sin 2\Theta\,
  \big(
  \frac{2\pi}{c}
  \big)^{\frac12}
  \sin^2\Phi \,J_{\frac52}(b) \\\nonumber
  = \,&
  -\pi\sin 2\Theta\,
  \sqrt{\frac1{kr}}\:
  \sin^2\Phi \,J_{\frac52}(2\pi kr) 
\end{align}
Therefore,
\begin{align}
  [\,r_1 r_2\,G^c(r)\,]^{\wedge} 
  =
  -\int_{r_c}^\infty
  \d dr\,
  r^4G(r)\,\pi\,
  \sqrt{\frac1{kr}}\:
  \sin^2\Phi\sin 2\Theta\,J_{\frac52}(2\pi kr)\,
\end{align}





For $\alpha = 1,\ \beta = 3$,
\begin{align}\nonumber
  [\,r_1 r_3\,G^c(r)\,]^{\wedge} 
  &= \int_{\mathbb R^3} r_1 r_3\,G^c(r)\, e^{-2\pi i\v k\cdot\v r}\d d\v r \\\nonumber
  &= \int_{r_c}^\infty \d dr \int_{0}^{2\pi} \d d\theta \int_0^\pi\d d\phi\:
  r^2\sin\phi \,r^2 \sin\phi\cos\theta\cos\phi  \,G(r)\,
  e^{-2\pi k r[\sin\phi\sin\Phi\cos(\theta - \Theta) + \cos\phi\cos\Phi]} \\
  &= \int_{r_c}^\infty \d dr \int_{0}^{2\pi} \d d\theta \int_0^\pi\d d\phi\:
  r^4\sin^2\phi\cos\phi\,G(r)\,e^{-2\pi i kr\cos\phi\cos\Phi}
  \cos\theta\,e^{-2\pi i kr\sin\phi\sin\Phi\cos(\theta- \Theta)}
\end{align}
Since
\begin{align}\nonumber
  \int_0^{2\pi}\cos\theta e^{-c i\cos(\theta - \Theta)} \d d\theta
  &=
  \int_0^{2\pi}\cos(t+\Theta) e^{-c i\cos t} \d d t\\\nonumber
  & =
  \int_0^{2\pi}
  \big[
   \cos t\cos \Theta - \sin t\sin \Theta
  \,\big]
  e^{-c i\cos t} \d dt \\\nonumber
  &=
  2\pi i  \cos \Theta J_1(-2\pi kr\sin\phi\sin\Phi)
\end{align}
And
\begin{align}\nonumber
  &
  \int_0^\pi
  2\pi i \cos \Theta\,
  \sin^2\phi\cos\phi\,
  J_1(b\sin\phi\sin\Phi) e^{b\cos\phi\cos\Phi} \,\d d\phi \\\nonumber
  = \,&
  \int_0^\pi
  2\pi i \cos \Theta\,
  \frac13\sin^2\phi\, C_1^{\frac32}(\cos\phi)
  J_1(b\sin\phi\sin\Phi) e^{b\cos\phi\cos\Phi} \,\d d\phi \\\nonumber
  = \,&
  2\pi i \cos \Theta\,
  \big(
  \frac{2\pi}{c}
  \big)^{\frac12}
  \sin\Phi\, \frac 13\,i\, C_1^{\frac32}(\cos\Phi)
  J_{\frac52}(b) \\\nonumber
  = \,&
  -2 \pi\,
  \sqrt{\frac1{kr}}\:
  \sin\Phi\cos\Phi\cos\Theta \,J_{\frac52}(2\pi kr) 
\end{align}
Therefore,
\begin{align}
  [\,r_1 r_3\,G^c(r)\,]^{\wedge} 
  =
  -\int_{r_c}^\infty
  \d dr\,
  r^4G(r)\,\pi\,
  \sqrt{\frac1{kr}}\:
  \sin2\Phi\cos\Theta\,J_{\frac52}(2\pi kr)\,
\end{align}
Similarly,
\begin{align}
  [\,r_2 r_3\,G^c(r)\,]^{\wedge} 
  =
  -\int_{r_c}^\infty
  \d dr\,
  r^4G(r)\,\pi\,
  \sqrt{\frac1{kr}}\:
  \sin2\Phi\sin\Theta\,J_{\frac52}(2\pi kr)\,
\end{align}





\newpage




\section{Results}






\begin{table}
  \centering
  \begin{tabular*}{0.95\textwidth}{@{\extracolsep{\fill}}c|l|ccccccc}\hline\hline
    $T^\ast$ & \textrm{method} &$r^\ast_{c}$ & $\mathcal E^\ast_{\textrm{target}}$  & $\rho^\ast_{\textrm{liq}}$ & $\rho^\ast_{\textrm{vap}}$ & $\gamma^\ast_{\textrm{sim}}$ & $\gamma^\ast_{\textrm{tail}}$ & $\gamma^\ast$ \\\hline
    & \textrm{unifom} & 5.0     & -       & 0.0024 (0) & 0.8364 (1) & 1.001 (7) & 0.120 & 1.121 (7) \\
    & \textrm{adapt}  & -       & 0.030   & 0.0024 (0) & 0.8358 (1) & 0.999 (9) & 0.138 & 1.137 (9) \\
0.70& \textrm{unifom} & 7.5     & -       & 0.0022 (0) & 0.8405 (1) & 1.081 (8) & 0.054 & 1.135 (8) \\
    & \textrm{adapt}  & -       & 0.006   & 0.0021 (0) & 0.8405 (1) & 1.098 (9) & 0.063 & 1.161 (9) \\
    & \textrm{ref}    & -       & -       & 0.0020 (0) & 0.8414 (1) & 1.081(11) & -     & 1.141(11) \\    \hline
    & \textrm{unifom} & 5.0     & -       & 0.0112 (1) & 0.7681 (2) & 0.707 (7) & 0.094 & 0.801 (7) \\
    & \textrm{adapt}  & -       & 0.030   & 0.0115 (1) & 0.7665 (2) & 0.673 (5) & 0.116 & 0.789 (5) \\
0.85& \textrm{unifom} & 7.5     & -       & 0.0102 (1) & 0.7741 (2) & 0.776 (7) & 0.044 & 0.819 (7) \\
    & \textrm{adapt}  & -       & 0.006   & 0.0101 (1) & 0.7737 (2) & 0.752(10) & 0.054 & 0.806(10) \\
    & \textrm{ref}    & -       & -       & 0.0098 (0) & 0.7750 (1) & 0.818(15) & -     & 0.818(15) \\    \hline
    & \textrm{unifom} & 5.0     & -       & 0.0635 (4) & 0.6258 (6) & 0.234 (8) & 0.044 & 0.279 (8) \\
    & \textrm{adapt}  & -       & 0.020   & 0.0667 (4) & 0.6219 (5) & 0.207(10) & 0.053 & 0.260(10) \\
1.10& \textrm{unifom} & 7.5     & -       & 0.0573 (4) & 0.6373 (5) & 0.281 (6) & 0.023 & 0.303 (6) \\
    & \textrm{adapt}  & -       & 0.004   & 0.0594 (4) & 0.6345 (5) & 0.275 (6) & 0.054 & 0.303 (6) \\
    & \textrm{ref}    & -       & -       & 0.0553 (0) & 0.6383 (1) & 0.302(11) & -     & 0.302(11) \\    \hline
    & \textrm{unifom} & 5.0     & -       & 0.1187 (6) & 0.5376(10) & 0.082 (6) & 0.019 & 0.102 (6) \\
    & \textrm{adapt}  & -       & 0.015   & 0.1391 (8) & 0.5262(16) & 0.054 (8) & 0.021 & 0.075 (8) \\
1.20& \textrm{unifom} & 7.5     & -       & 0.1057 (7) & 0.5588(10) & 0.120 (8) & 0.012 & 0.132 (8) \\
    & \textrm{adapt}  & -       & 0.003   & 0.1063 (9) & 0.5550 (8) & 0.113 (5) & 0.015 & 0.129 (5) \\
    & \textrm{ref}    & -       & -       & 0.0942 (0) & 0.5627 (2) & 0.156(14) & -     & 0.156(14) \\    \hline    \hline    
  \end{tabular*}
\end{table}


% \newpage

% \bibliography{ref}{}
% \bibliographystyle{unsrt}


\end{document}

\documentclass{letter}

\usepackage{graphicx}

\renewcommand{\baselinestretch}{1.0}
% \setlength{\footskip}{0cm}
%\addtolength{\textheight}{10cm}
% \sloppy

\address{Institute for Mathematics \\
  Freie Universit\"at Berlin\\
  Arnimallee 6\\
  14195 Berlin, Germany 
}
\signature{Han Wang}

% \def\urltilda{\kern -.15em\lower .7ex\hbox{\~{}}\kern .04em}
% \def\urlunder{\lower .1ex\hbox{\_{}}\kern .04em}
% \def\urldot{\kern -.10em.\kern -.10em}
% \def\urlhttp{http\kern -.10em\lower -.1ex\hbox{:}\kern -.12em\lower 0ex\hbox{/}\kern -.18em\lower 0ex\hbox{/}}
% \parindent0ex


\begin{document}

\begin{letter}{
    The Editors of\\
    Physical Review E\\
    \vskip 1cm
  }
% \begin{picture}(0,0)
%   \put(-30,22){
%     \parbox{0.25\textwidth}{
% %	\includegraphics[scale=0.4]{logo-icp-notext.pdf}
%     }
%   }
%   \put(264,10){
%     \parbox{0.25\textwidth}{
% 	\includegraphics[scale=0.28]{Fub-logo.eps}
%     }
%   }
% \end{picture}

\opening{Dear Editors,\vskip .5cm}
Hereby we would like to submit our article:\\

\textbf{The Error Estimate of Force Calculation in the Inhomogeneous Molecular Systems}\\

by H. Wang, C. Sch\"utte and P. Zhang to:\\

\textbf{Physical Review E}, as a regular article.\\

The molecular dynamics simulation of the inhomogeneous molecular
systems, in which the density of a certain type of atom is lack of
uniformity, will encounter non-trivial difficulties by using the
standard cut-off method. The simulation results are found to converge
only at a very large cut-off radius, so it is computationally very
expensive to obtain precise measurement of the physical properties.\\

To quantitatively study the cut-off artifact, we developed an accurate
error estimate of the standard cut-off method for the inhomogeneous
systems. We proved that the root mean square force error is decomposed
into three additive parts: the homogeneity error, the inhomogeneity
error and the correlation error.  An important observation is that the
inhomogeneity error is dominant in the interfacial regions where the
density changes rapidly, and can be more than one order of magnitude
larger than the error in the bulk regions where the density is
comparatively uniform.  Therefore, a large cut-off value should be
used to suppress the maximum error in the interfacial regions, but at
the mean time,
considerable computational effort is wasted in the bulk regions. \\

We proposed the adaptive cut-off and the force correction method to
avoid the mentioned problem. The adaptive cut-off method uses different
cut-off radii for the interfacial and bulk regions so that the error
is uniformly distributed across the system. The force correction
method removes the inhomogeneity error, therefore, the precision of
force calculation is improved, so a smaller cut-off radius is
acceptable. Both of the proposed methods relies on the quantitative
analysis of the short-range force error.  The effectiveness of
the proposed methods were checked by molecular dynamics simulations of
the liquid-vapor equilibrium and the nanoscale
particle collision.\\

Most of previous error estimates in this field assumed the homogeneity
of the system. As far as we know, the present work for the first time
proposed an error estimate for the short-range interactions in the
inhomogeneous molecular systems, presenting new phenomena that do not
happen in the homogeneous systems.  We showed that the adaptive cut-off
and the force correction methods greatly improve the efficiency and
accuracy, and successfully avoid the problem of the standard cut-off
method, so they should be of special interest for
understanding the inhomogeneous molecular systems, and for
designing high
performance molecular simulation softwares.
\\
\vskip 3cm

\closing{Best regards,}
  % \put(-85,-50){\includegraphics[scale=0.2]{zhang-signature.pdf}}
% }


\end{letter} \end{document}

\documentclass{scrlttr2}
\usepackage{graphicx}
\usepackage{hyperref}
% provides a justified text without hyphenation
\usepackage[none]{hyphenat}
\renewcommand{\baselinestretch}{1.0}
\setlength{\footskip}{0cm}
%\addtolength{\textheight}{10cm}
\sloppy


\KOMAoptions{foldmarks=false,fromalign=left,fromrule=true,fromlogo=false,backaddress=false}
\setkomavar{fromname}{Dr. Han Wang}
\setkomavar{fromaddress}
{Department of Mathematics \& Computer Science \\
  Freie Universit\"at Berlin\\
  Arnimallee 6\\
  14195 Berlin, Germany 
}

\def\urltilda{\kern -.15em\lower .7ex\hbox{\~{}}\kern .04em}
\def\urlunder{\lower .1ex\hbox{\_{}}\kern .04em}
\def\urldot{\kern -.10em.\kern -.10em}
\def\urlhttp{http\kern -.10em\lower -.1ex\hbox{:}\kern -.12em\lower 0ex\hbox{/}\kern -.18em\lower 0ex\hbox{/}}
\parindent0ex


\begin{document}
\makeatletter
\@setplength{toaddrvpos}{6em}
\@setplength{sigbeforevskip}{0.5em}
\@setplength{sigindent}{0em}
\@setplength{refvpos}{35ex}

\begin{letter}{
    The Referees of\\
    Physical Chemistry Chemical Physics\\
    \vskip 0cm
  }
\begin{picture}(0,0)
  \put(-30,22){
    \parbox{0.25\textwidth}{
%	\includegraphics[scale=0.4]{logo-icp-notext.pdf}
    }
  }
  \put(264,10){
    \parbox{0.25\textwidth}{
	\includegraphics[scale=0.28]{Fub-logo.pdf}
    }
  }
\end{picture}

\opening{Dear Referees,}
Hereby we would like to submit our article:\\

\textbf{The Error Estimate of Force Calculation in the Inhomogeneous Molecular Systems}\\

by H. Wang, Christof Sch\"utte and P. Zhang to:\\

\textbf{Physical Chemistry Chemical Physics}, as a regular article.\\

The molecular dynamics simulation of the inhomogeneous molecular
systems, in which the density of a certain type of atom is lack of
uniformity, will encounter non-trivial difficulties by using the
standard cut-off method. The simulation results are found to converge
only at a very large cut-off radius, so it is computationally very
expensive to obtain precise measurement of the physical properties.\\

To quantitatively study the cut-off artifact, we developed an accurate
error estimate of the standard cut-off method for the inhomogeneous
systems. We proved that the root mean square force error is decomposed
into three additive parts: the homogeneity error, the inhomogeneity
error and the correlation error.  An important observation is that the
inhomogeneity error is dominant in the interfacial regions where the
density changes rapidly, and can be more than one order of magnitude
larger than the error in the bulk regions where the density is
comparatively uniform.  Therefore, a large cut-off value should be
used to suppress the maximum error in the interfacial regions, but at
the mean time,
considerable computational effort is wasted in the bulk regions. \\

We proposed the adaptive cut-off and the force correction method to
avoid the mentioned problem. The adaptive cut-off method uses different
cut-off radii for the interfacial and bulk regions so that the error
is uniformly distributed across the system. The force correction
method removes the inhomogeneity error, therefore, the precision of
force calculation is improved, so a smaller cut-off radius is
acceptable. Both of the proposed methods relies on the quantitative
analysis of the short-range force error.  The effectiveness of
the proposed methods were checked by molecular dynamics simulations of
the liquid-vapor equilibrium and the nanoscale
particle collision.\\

Most of previous error estimates in this field assumed the homogeneity
of the system. As far as we know, the present work for the first time
proposed an error estimate for the short-range interactions in the
inhomogeneous molecular systems, presenting new phenomena that do not
happen in the homogeneous systems.  We shew that the adaptive cut-off
and the force correction methods greatly improve the efficiency and
accuracy, and successfully avoid the problem of the standard cut-off
method, so they should be of special interest for designing high
performance molecular simulation software.
\\

\closing{Best regards,
  % \put(-85,-50){\includegraphics[scale=0.2]{zhang-signature.pdf}}
}


% \newpage

% \parindent0ex \textbf{Corresponding author}:\\

% \qquad Pingwen Zhang

% \qquad LMAM and School of Mathematical Sciences

% \qquad Peking University

% \qquad 100871 Beijing, P.R. China

% \qquad Phone: +86-10-62759851

% \qquad Fax: +86-10-62751801

% \qquad \href{mailto:pzhang@pku.edu.cn}{\tt pzhang@pku.edu.cn} \\


% \parindent0ex\textbf{Referees}:\\
% Furthermore as knowledgeable referees in this field we suggest:
% \begin{itemize}
% \item Christoph Dellago \\
%   Faculty of Physics\\
%   University of Vienna\\
%   Boltzmanngasse 5 (1st floor, room 3105) \\
%   1090 Vienna\\
%   Austria\\
%   \href{mailto:Christoph.Dellago@univie.ac.at}{\tt Christoph.Dellago@univie.ac.at}
% \item Giovanni Ciccotti \\
%   Department of Physics \\
%   University of Roma "La Sapienza"\\
%   P.le Aldo Moro 5\\
%   00185 Roma\\
%   Italy\\
%   \href{mailto:giovanni.ciccotti@phys.uniroma1.it}{\tt giovanni.ciccotti@phys.uniroma1.it}
% \item Daan Frenkel\\
%   Cambridge University Centre for Computational Chemistry\\
%   Department of Chemistry\\
%   Lensfield Road\\
%   Cambridge CB2 1EW\\
%   United Kindom\\
%   \href{mailto:df246@cam.ac.uk}{\tt df246@cam.ac.uk}
% \item Kurt Binder\\
%   KOMET 331 \\
%   Institute of Physics\\
%   Johannes Gutenberg University Mainz \\
%   D 55099 Mainz \\
%   Germany\\
%   \href{mailto:kurt.binder@uni-mainz.de}{\tt kurt.binder@uni-mainz.de}
% \end{itemize}

% \parindent0ex\textbf{Color figures}:
% \begin{itemize}
%   \item[] A main aspect of the article is accuracy, therefore even smallest
%     deviations should be recognizable in the plots. Though we already used
%     different line styles, coloring is inevitable to distinguish the various
%     amount of information correctly. Therefore we regard colored versions of
%     Figure 1 and 2 as essential to provide a clear picture for the reader.
% \end{itemize}


\end{letter} \end{document}

\documentclass[a4paper]{article}

\begin{document}

\noindent
{Dear Editors,}\\

Hereby we would like to resubmit our article
(Manuscript No. ES10751):\\

\textbf{The Error Estimate of Short-Range Force Calculation in the Inhomogeneous Molecular Systems
},\\

by H. Wang, C. Sch\"utte and P. Zhang to:\\

\textbf{Physical Review E}, as a regular article.\\

We would like to thank the referees for carefully reading our
manuscript and the comments helping us to finally provide a clearer
and more valuable article.  Some changes to the manuscript have been
done, as suggested by the
referees, and are explained in the following.\\

\textbf{Referee 2}\\

\textit{I also did not understand how the adaptive cutoff method results were
plotted as a function of cutoff distance in Figs 5 \& 6. A variety of
cutoffs are being used, yet it is plotted against a single cutoff
distance?}

Throughout the paper, when we mention
that ``the cut-off of
the adaptive cut-off method is $r_c^\ast$'', we mean that the
control error  $\mathcal E^\ast_{\textrm{\tiny{C}}}$
is the same as the maximum error of a uniform cut-off simulation
using $r_c^\ast$.
Under this setting,  the maximum 
cut-off radius of the adaptive cut-off method is the same as,
or sometimes marginally larger  than
$r_c^\ast$ (no more than $r_c^{\textrm{\tiny{step}}}$).
We have changed the captions of Figures 5 and 6 to make it clear.
\\

\textit{
I also suggest you make brief mention of the following issues, and
explain how your methods deal with them:
}

\textit{
(1) Does the ARC method (adaptive cut-off) make sense for long-range
Coulombic calculations where there is a short-range part with a cutoff
coupled to a long-range Kspace calculation (e.g. particle-mesh Ewald).
This is a commonly used approach for charged molecular systems (e.g.
solvated biomolecules). In that case, I believe the Kspace formulation
depends on their being a single cutoff on the short-range portion.
}

We  have recently submitted a paper,
titled ``The Numerical Accuracy of Computing
Electrostatic Interaction in the Inhomogeneous and Correlated
Molecular Systems: for the Ewald Summation, SPME and Staggered Mesh
Ewald Methods''.
In that paper, the error estimates for both the
direct space short-range part and the reciprocal space long-range
part are developed.
We prove that the inhomogeneity error vanishes in the
locally neutral systems, so
there is no error anomaly at the interfacial regions. Since
most practical charged systems are locally neutral,
we are basically safe and
do not need neither ARC (adaptive cut-off)
nor LFC (long-range force correction)
for the electrostatic calculation. We 
have also tested an artificial system that is NOT locally neutral. We find
the direct space error presents similar error anomaly at the interface,
while the reciprocal error (Kspace error) does not
have any anomaly.
Therefore, the reciprocal force calculation
needs no adjustment, so the reciprocal space working parameters
(interpolation order and grid spacing) and the splitting
parameter can be fixed. When these parameters are fixed, the direct
space force calculation is decoupled from the reciprocal space,
so the ARC can be applied.
\\

\textit{
(2) Can the spatial error estimation technique and 2 correction
methods be used for either periodic or non-periodic systems?
}

As far as we concerned, the error estimate technique and proposed
corrections can only be used to periodic systems.
We have added the following sentence to stress this point
(line 8, page 26):
\textbf{It should be noticed that the proposed error estimate and correction
methods can only be used for periodic systems.}
\\


\textit{
  (3) You say that the additional cost of the methods is small even
though they involve FFTs that scale as $N\log N$, compared to the $O(N)$
scaling of short-range MD. Your timing results bear this out, however
these examples were all fairly small systems. If large simulations are
performed (e.g. multi-million atoms), is the extra cost still small?}

This is a very important comment, because
the cost of FFTs ($O(N\log N)$) will overcome the cost of short-range
interaction ($O(N)$) at some point when the system is very
large. Eventually the computationally cost will be overwhelmed by the
FFTs as the size of the system grows to infinity.
This point should be noticed.  Fortunately, for the systems
also calculate long-range interactions, the short-range error estimate
scales the same as the Ewald type long-range algorithms.
We have added the following remarks to the paper
(2nd line from the bottom, page 26):
\textbf{In the systems tested by the present paper, the computational
  load of the error estimate is small compared to that of the
  short-range interaction.
  However, it should be noticed that
  the cost of the error estimate scales as $\mathcal O(N\log N)$.
  If the system only has short-range interaction that
  scales as $\mathcal O(N)$, the load of the error estimate
  may become comparatively expensive when the system is large,
  and eventually
  overwhelms the total computational cost as the
  system size grows to infinity.
  If the system also needs long-range electrostatic calculation
  that is usually treated by the $\mathcal O(N\log N)$ scaling
  Ewald type algorithms [22-24],
  the extra load of the error estimate is likely to be acceptable.
}


\end{document}

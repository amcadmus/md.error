\documentclass[aps, pre, preprint,unsortedaddress,a4paper,onecolumn]{revtex4}
% \documentclass[aps, pre, preprint,unsortedaddress,a4paper,twocolumn]{revtex4}
% \documentclass[acs, jctcce, a4paper,preprint,unsortedaddress,onecolumn]{revtex4-1}
% \documentclass[aps,pre,twocolumn,unsortedaddress]{revtex4-1}
% \documentclass[aps,jcp,groupedaddress,twocolumn,unsortedaddress]{revtex4}

\usepackage[fleqn]{amsmath}
\usepackage{amssymb}
\usepackage[dvips]{graphicx}
\usepackage{color}
\usepackage{tabularx}
\usepackage{algorithm}
\usepackage{algorithmic}

\makeatletter
\makeatother

\newcommand{\recheck}[1]{{\color{red} #1}}
\newcommand{\redc}[1]{{\color{red} #1}}
\newcommand{\bluec}[1]{{\color{red} #1}}
\newcommand{\greenc}[1]{{\color{green} #1}}
\newcommand{\vect}[1]{\textbf{\textit{#1}}}
% \newcommand{\vect}[1]{#1}
\newcommand{\dd}[1]{\textsf{#1}}
% \newcommand{\fwd}[0]{\textrm{fw}}
% \newcommand{\bwd}[0]{\textrm{bw}}
\newcommand{\fwd}[0]{+}
\newcommand{\bwd}[0]{-}
\newcommand{\period}[0]{T_{\textrm{P}}}
\newcommand{\ml}[0]{\mathcal {L}}
\newcommand{\mo}[0]{\mathcal {O}}
\newcommand{\mbp}[0]{\mathbb {P}}
\newcommand{\mc}[0]{\mathcal {C}}
\newcommand{\dt}[0]{\Delta t}
\newcommand{\id}{\mathrm{Id}}
\newcommand{\mypi}{\boldsymbol\pi}
% \newcommand{\mymu}{\boldsymbol\mu}
\newcommand{\myphi}{\Phi}
\newcommand{\mymu}{\mu}
\newcommand{\prob}{\textrm{P}}
\newcommand{\ins}{\textrm{ins}}
\newcommand{\target}{\textrm{ref}}
\newcommand{\trace}{\textrm{Tr}}
\newcommand{\dvs}{\dot{\vect s}}
\newcommand{\dvh}{\dot{\vect h}}
\newcommand{\dvr}{\dot{\vect r}}
\newcommand{\dvp}{\dot{\vect p}}
\newcommand{\inv}{{-1}}

\newcommand{\confaa}[0]{{\alpha_{\textrm{R}}}}
\newcommand{\confab}[0]{{\alpha_{\textrm{R}}'}}
\newcommand{\confba}[0]{{\textrm{C}7_{\textrm{eq}}}}
\newcommand{\confbb}[0]{{\textrm{C}5}}
\newcommand{\confc}[0]{{\alpha_{\textrm{L}}}}
\newcommand{\temp}[0]{{\textrm{T}}}
\newcommand{\pres}[0]{{\textrm{P}}}



\begin{document}

\title{Structure preserving numerical scheme for isothermal isobaric Langevin dynamics}
\author{Han Wang}
\email{wang_han@iapcm.ac.cn}
\affiliation{CAEP, Beijing, China}
   
\begin{abstract}
\end{abstract}

\maketitle

\section{Theoretical background: Isotropic case}
\subsection{First Hamiltonian definition}

The isobaric simulation is implmented by rescaling the simulation region.
Here we assume the box is cubic and of volume $V$. The positions of the
atoms are denoted by $\vect r_i,\ i=1, \cdots, N$. 
The rescaling is given by
\begin{align}
  \vect s_i = V^{-\frac13} \vect r_i, \quad \vect r_i = V^{\frac13}\vect s_i
\end{align}
Therefore, the straightforward definition of the kinetic energy is given by
\begin{align}
  \mathcal K &=
  \sum_{i=1}^N \frac12 m_i \big(\frac{d}{dt} (V^{\frac13}\vect s_i) \big)^2
  +
  \frac 12 {M_V} \dot V^2 \\
  & =
  \sum_{i=1}^N \frac 12 m_i
  \big(\frac13 V^{-\frac23} \dot V \vect s_i + V^{\frac13}\dot{\vect s}_i\big)^2
  +
  \frac 12 {M_V} \dot V^2 
\end{align}
The corresponding momenta are
\begin{align}
  \mypi_i
  &=
  \frac{\partial \mathcal K}{\partial \dot{\vect s}_i} =
  m_i \big(\frac13 V^{-\frac23} \dot V \vect s_i + V^{\frac13}\dot{\vect s}_i\big) V^{\frac13}  \\
  p_V
  &=
  \frac{\partial \mathcal K}{\partial \dot{V}}
  =
  \sum_{i=1}^N m_i
  \big(\frac13 V^{-\frac23} \dot V \vect s_i + V^{\frac13}\dot{\vect s}_i\big)\,
  \frac13 V^{-\frac23}\vect s_i
  +
  {M_V}\dot V \\
  &=
  \sum_{i=1}^N 
  \frac13 V^{-1} \mypi_i\cdot\vect s_i + {M_V} \dot V
\end{align}
The Hamiltonian of the system is therefore given by
\begin{align}
  \mathcal H =
  \sum_{i=1}^N \mypi\cdot\dot{\vect s}_i + p_V \dot V - \mathcal K + \mathcal V,
\end{align}
where $\mathcal V$ is the potential energy:
\begin{align}
  \mathcal V = U(V^{\frac13}\vect s_1, \cdots, V^{\frac13}\vect s_N) + PV
\end{align}
then
\begin{align}\nonumber
  \mathcal H &=
  \sum_{i=1}^N
  m_i \big(\frac13 V^{-\frac23} \dot V \vect s_i + V^{\frac13}\dot{\vect s}_i\big) V^{\frac13}\dot{\vect s}_i
  + 
  \sum_{i=1}^N
  m_i \big(\frac13 V^{-\frac23} \dot V \vect s_i + V^{\frac13}\dot{\vect s}_i\big)
  \frac13 V^{-\frac23}\vect s_i \dot V
  - \mathcal K + \mathcal V \\
  & =
  \mathcal K + \mathcal V \\
  &=
  \sum_{i=1}^N \frac 12 m_i
  \big(\frac13 V^{-\frac23} \dot V \vect s_i + V^{\frac13}\dot{\vect s}_i\big)^2
  +
  \frac 12 {M_V} \dot V^2
  +
  U(V^{\frac13}\vect s_1, \cdots, V^{\frac13}\vect s_N) + PV \\
  &=
  \sum_{i=1}^N\frac{V^{-\frac23} \mypi_i^2}{2m_i}
  +
  \frac1{2{M_V}}\big(p_V - \sum_{i=1}^N\frac13V^{-1} \mypi_i\cdot\vect s_i\big)^2
  +
  U(V^{\frac13}\vect s_1, \cdots, V^{\frac13}\vect s_N) + PV 
\end{align}
The equation of motion:
\begin{align}
  \dot{\vect s}_i
  &=
  \frac{\partial \mathcal H}{\partial \mypi_i}
  =
  \frac{V^{-\frac23}\mypi_i}{m_i} -
  \frac1{{M_V}}\big(p_V - \sum_{i=1}^N\frac13V^{-1} \mypi_i\cdot\vect s_i\big)\frac13 V^{-1}\vect s_i \\
  \dot{\mypi}_i
  &= 
  -\frac{\partial \mathcal H}{\partial \vect s_i}
  = -\partial_i U V^{\frac13} +
  \frac1{{M_V}}\big(p_V - \sum_{i=1}^N\frac13V^{-1} \mypi_i\cdot\vect s_i\big)\frac13 V^{-1}\mypi_i \\
  \dot{V} 
  &=
  \frac{\partial \mathcal H}{\partial \vect p_V}
  =\frac1{M_V}\big(p_V - \sum_{i=1}^N\frac13V^{-1} \mypi_i\cdot\vect s_i\big)\\
  \dot{p}_V
  &=
  -\frac{\partial \mathcal H}{\partial V}
  =
  \sum_{i=1}^N \frac13V^{-\frac53} \frac{ \mypi_i^2}{m_i}
  -
  \frac1{{M_V}}\big(p_V - \sum_{i=1}^N\frac13V^{-1} \mypi_i\cdot\vect s_i\big)
  \sum_{i=1}^N\frac13V^{-2} \mypi_i\cdot\vect s_i
  -
  \sum_{i=1}^N\partial_i U \frac13V^{-\frac23}\vect s_i - P
\end{align}
These quations of motion are too complicated.  It seems there is no good way to simplify it.

\subsection{Second Hamiltonian definition}
We use an alternative way of defining the kinetic energy:
\begin{align}
  \mathcal K &=
  \sum_{i=1}^N \frac12 m_i V^{\frac23}\dot{\vect s}_i^2
  +
  \frac 12 {M_V} \dot V^2 
\end{align}
The momenta are given by
\begin{align}
  \mypi_i
  &=
  \frac{\partial \mathcal K}{\partial\dot{\vect s}_i}
  =m_i V^{\frac23}\dot{\vect s}_i\\
  p_V
  &=
  \frac{\partial \mathcal K}{\partial\dot{V}}
  = {M_V}\dot V
\end{align}
The the Hamiltonian is
\begin{align}
  \mathcal H
  &=
  \sum_{i=1}^N\mypi_i\cdot \dot{\vect s}_i +
  p_V \dot V - \mathcal K + \mathcal V \\
  & =
  \mathcal K + \mathcal V \\
  &=
  \sum_{i=1}^N \frac12 m_i V^{\frac23}\dot{\vect s}_i^2
  +
  \frac 12 {M_V} \dot V^2  +
  U(V^{\frac13}\vect s_1, \cdots, V^{\frac13}\vect s_N) + PV    \\
  & =
  \sum_{i=1}^N V^{-\frac23} \frac{\mypi_i^2}{2m_i}
  +
  \frac{p_V^2}{2{M_V}}
  +
  U(V^{\frac13}\vect s_1, \cdots, V^{\frac13}\vect s_N) + PV      
\end{align}
The corresponding equations of motion are:
\begin{align}
  \dot{\vect s}_i
  &=
  \frac{\partial \mathcal H}{\partial \mypi_i}
  =
  V^{-\frac23} \frac{\mypi_i}{m_i} \\
  \dot{\mypi}_i
  &=
  -\frac{\partial \mathcal H}{\partial \vect s_i}
  = -\partial_i U V^{\frac13} \\
  \dot{V} 
  &=
  \frac{\partial \mathcal H}{\partial p_V}
  = \frac{p_V}{M_V} \\
  \dot{p}_V
  &=
  -\frac{\partial \mathcal H}{\partial V}
  =
  \sum_{i=1}^N \frac13 V^{-\frac53} \frac{\mypi_i^2}{m_i}
  -
  \sum_{i=1}^N \partial_i U \frac13 V^{-\frac23} \vect s_i
  - P
\end{align}

\subsection{Generalized Langevin equation}
we want to write the equations of motion in terms of the generlized Langevin quations:
\begin{align}
  \dot{\vect s}_i
  &=
  \frac{\partial \mathcal H}{\partial \mypi_i} \\
  \dot{\mypi}_i
  &=
  -\frac{\partial \mathcal H}{\partial \vect s_i} - \gamma_i\frac{\partial H}{\partial\mypi_i} + \sigma_i \dot W\\
  \dot{V} 
  &=
  \frac{\partial \mathcal H}{\partial p_V}\\
  \dot{p}_V
  &=
  -\frac{\partial \mathcal H}{\partial V}
  -\gamma_V\frac{\partial \mathcal H}{\partial p_V} + \sigma_V \dot W
\end{align}
Equivalently:
\begin{align} \label{eq:lang-00}
  \dot{\vect s}_i
  &=
  V^{-\frac23} \frac{\mypi_i}{m_i} \\ \label{eq:lang-01}
  \dot{\mypi}_i
  &=
  -\partial_i U V^{\frac13} - \gamma_iV^{-\frac23} \frac{\mypi_i}{m_i} + \sigma_i\dot W \\\label{eq:lang-02}
  \dot{V} 
  &=
  \frac{p_V}{M_V} \\\label{eq:lang-03}
  \dot{p}_V
  &=
  \sum_{i=1}^N \frac13 V^{-\frac53} \frac{\mypi_i^2}{m_i}
  -
  \sum_{i=1}^N \partial_i U \frac13 V^{-\frac23} \vect s_i
  - P - \gamma_V  \frac{p_V}{M_V} + \sigma_V\dot W
\end{align}
Noticing that $\vect S$ and $\mypi$ are not physical variable, we do the
following variable transformations:
\begin{align}
  \vect s
  &= V^{-\frac13}\vect r \\
  \dot{\vect s}
  &=
  V^{-\frac13}\dot{\vect r} - \frac13 V^{-\frac43}\dot V\vect r\\
  \mypi
  &= V^{\frac13}\vect p\\
  \dot\mypi
  &=
  V^{\frac13}\dot{\vect p} + \frac13 V^{-\frac23}\dot V\vect p
\end{align}
Then Eq.~\eqref{eq:lang-00}--\eqref{eq:lang-03} become
\begin{align}
  \dot{\vect r}_i
  &= \frac{\vect p_i}{m_i} + \frac{\dot V}{3V} \vect r_i\\
  \dot{\vect p}_i
  & =
  -\partial_i U - \frac{\dot V}{3V}\vect p_i - \gamma_iV^{-\frac23}\frac{\vect p_i}{m_i}
  + \sigma_iV^{-\frac13}\dot W\\
  \dot{V} 
  &=
  \frac{p_V}{M_V} \\
  \dot{p}_V
  &=
  \sum_{i=1}^N \frac13 V^{-1} \frac{\vect p_i^2}{m_i}
  -
  \sum_{i=1}^N \frac13 V^{-1} \partial_i U  \vect r_i
  - P - \gamma_V  \frac{p_V}{M_V} + \sigma_V\dot W 
\end{align}
We denote
\begin{align}
  \hat \gamma_i = \gamma_i V^{-\frac23}, \quad \hat \sigma_i = \sigma_i V^{-\frac13}
\end{align}
Here the new notation satisfies the same fluctuation-disspation relation as the old one.
The instantaneous pressure is given by
\begin{align}
  P =
  \sum_{i=1}^N \frac13 V^{-1} \frac{\vect p_i^2}{m_i}
  -
  \sum_{i=1}^N \frac13 V^{-1} \partial_i U  \vect r_i
\end{align}
Therefore we end up with the Langevin equations in the transformed variables:
\begin{align}
  \dot{\vect r}_i
  &= \frac{\vect p_i}{m_i} + \frac{p_V}{3M_VV} \vect r_i\\
  \dot{\vect p}_i
  & =
  -\partial_i U - \frac{p_V}{3M_VV}\vect p_i - \hat\gamma_i\frac{\vect p_i}{m_i}
  + \hat\sigma_i\dot W\\
  \dot{V} 
  &=
  \frac{p_V}{M_V} \\
  \dot{p}_V
  &=
  P - P_\target - \gamma_V  \frac{p_V}{M_V} + \sigma_V\dot W   
\end{align}
The fluctuation-disspation theorem claims that:
\begin{align}
  \hat \sigma_i^2 = 2k_BT \hat\gamma_i ,\quad
  \sigma_V^2 = 2k_BT \gamma_V 
\end{align}
We set:
\begin{align}
  \hat\sigma_i^2 / m_i = \sigma_\temp^2, \quad \hat\gamma_i / m_i = \gamma_\temp,
\end{align}
and
\begin{align}
  \sigma_V^2 / M_V = \sigma_\pres^2, \quad \gamma_V / M_V = \gamma_\pres,
\end{align}
where $\gamma_\temp$, $\gamma_\temp$, $\sigma_\temp$ and $\sigma_\pres$ are constants, then
The fluctuation-disspation theorem becomes
\begin{align}
  \sigma_\temp^2 = 2k_BT \gamma_\temp ,\quad
  \sigma_\pres^2 = 2k_BT \gamma_\pres.
\end{align}
The equations of motion are
\begin{align}\label{eq:eom-1}
  \dot{\vect r}_i
  &= \frac{\vect p_i}{m_i} + \frac{p_V}{3M_VV} \vect r_i\\\label{eq:eom-2}
  \dot{\vect p}_i
  & =
  -\partial_i U - \frac{p_V}{3M_VV}\vect p_i - \gamma_\temp{\vect p_i}
  + m_i^{\frac12}\sigma_\temp\dot W\\ \label{eq:eom-3}
  \dot{V} 
  &=
  \frac{p_V}{M_V} \\\label{eq:eom-4}
  \dot{p}_V
  &=
  P - P_\target - \gamma_\pres  {p_V} +M_V^{\frac12} \sigma_\pres\dot W   
\end{align}
Please note that the variable $\vect p$ here is not exactly the same as the physical momentum $\tilde {\vect p}$:
\begin{align}\nonumber
  \tilde {\vect p}
  &= m \dot{\vect r}
  = m \frac{d}{dt}\big(V^{\frac13}\vect s\big)
  = m V^{\frac13}\dot{\vect s} + \frac13 V^{-\frac23}\dot Vm\vect s \\\nonumber
  & =
  V^{-\frac13}\mypi + \frac{\dot V}{3V}m V^{\frac13}\vect s \\
  &=
  \vect p + \frac{\dot V}{3V}m\vect r
\end{align}

\subsection{The choice of parameters}
\recheck{bug fix in FD theorem, may be some changes in this section, not checked}

Substituting Eq.~\eqref{eq:eom-3} into \eqref{eq:eom-4}, and neglecting the
Gaussian noise, we have
\begin{align}\label{eq:damp-v-0}
  M_V \ddot V = P - P_\target - \gamma_\pres M_V \dot V 
\end{align}
A first order taylor expansion of the instantaneous volume with respect to the
reference volume gives:
\begin{align}
  V - V_\target = -\kappa_\temp V_\target (P - P_\target),
\end{align}
where the isothermal compressibility is defined by
\begin{align}
  \kappa_\temp =-\frac 1V \Big(\frac{\partial V}{\partial P}\Big)_\temp
\end{align}
The Eq.~\eqref{eq:damp-v-0} becomes
\begin{align}\label{eq:damp-v-1}
  M_V \ddot V = -\frac{1}{\kappa_\temp V_\target}(V - V_\target) - \gamma_\pres M_V \dot V
\end{align}
This is comparable to the equation of motion of the damped harmonic oscillator:
\begin{align}
  m\ddot x = -kx - c\dot x
\end{align}
For the damped harmonic oscillator, the angular frequency is
\begin{align}
  \omega_0 = \sqrt{\frac km}
\end{align}
And the damping ratio is
\begin{align}
  \zeta = \frac{c}{2\sqrt{mk}}
\end{align}
Converting writing angular frequency and the damping ration in terms of the
Eq.~\eqref{eq:damp-v-1}, we have
\begin{align}
  \omega_0 = \frac{1}{\sqrt{M_V \kappa_\temp V_\target}}, \quad
  \zeta = \frac{\gamma_\pres}{2}\sqrt{M_V\kappa_\temp V_\target}
\end{align}
We denote the time-scale of pressure convergence by $\tau_\pres$, and ask for
\begin{align}
  \omega_0 = \frac{2\pi}{\tau_\pres}
\end{align}
Then
\begin{align}
  M_V = \frac{1}{\kappa_T V_\target}\Big(\frac{\tau_\pres}{2\pi}\Big)^2
\end{align}
We use the damping ratio of $\zeta = 1/(4\pi)$ leads to
\begin{align}
  \gamma_\pres = \frac{1}{\tau_\pres}
\end{align}
We do not know in priori the value of $V_\target$, but we can use the
inital volume $V_0$ as a good guess, if the system is not very far from
equilibrium.
Similiarly, we use 
\begin{align}
  \gamma_\temp = \frac{1}{\tau_\temp}.
\end{align}


\section{Anisotropic box shape fluctuation}

We define the affine transformation from the logic (reference) to the physical coordinates
of an atom by
\begin{align}
  \vect r_i = \vect h \vect s_i
\end{align}
Similar to the isotropic case, the kinetic energy is defined by
\begin{align}
  \mathcal K = \sum_i \frac12 m_i (\vect h \dvs_i)^\top(\vect h \dvs_i) + \frac12 M_h \trace (\dvh^\top \dvh)
\end{align}
The momenta are defined by
\begin{align}
  \mypi_i & = \frac{\partial\mathcal K}{\partial\dvs_i} = m_i \vect h^\top\vect h \dvs_i\\
  \vect p_h &= \frac{\partial \mathcal K}{\partial \dvh}  = M_h \dvh
\end{align}
We now write down the Hamiltonian of the system by:
\begin{align}\nonumber
  \mathcal H & = \sum_i \mypi_i\cdot\dvs_i + \trace (\vect p_h^\top \cdot \dvh ) - \mathcal K + U(\vect h\vect s_1, \cdots, \vect h\vect s_N) + P\det (\vect h)\\\nonumber
             & = \sum_i \frac12 m_i (\vect h \dvs_i)^\top(\vect h \dvs_i) + \frac12 M_h \trace (\dvh^\top \dvh) + U(\vect h\vect s_1, \cdots, \vect h\vect s_N) + P\det (\vect h)\\ \label{eqn:a-hamiltonian}
             & = \sum_i \frac1{2m_i} \mypi_i^\top \vect h^{-1}\vect h^{-\top}\mypi_i + \frac1{2M_h}\trace (\vect p_h^\top \vect p_h)  + U(\vect h\vect s_1, \cdots, \vect h\vect s_N) + P\det (\vect h)
\end{align}
The equations of motion are therefore
\begin{align}
  \dvs_i &= \frac{\partial\mathcal H}{\partial \mypi_i} = \frac1{m_i}\vect h^{-1}\vect h^{-\top}\mypi_i \\
  \dot{\mypi_i} & = -\frac{\partial \mathcal H}{\partial \vect s_i} =  - \vect h^\top \partial_i U\\
  \dvh &= \frac{\partial\mathcal H}{\partial \vect p_h} = \frac1{M_h}{\vect p_h}\\ \label{eqn:a-dph}
  \dot{\vect p}_h & = - \frac{\partial\mathcal H}{\partial \vect h}
\end{align}


We explain here the derivation of the R.H.S.~of \eqref{eqn:a-dph}. In
the Hamiltonian~\eqref{eqn:a-hamiltonian}, the first, third and fourth
terms are $\vect h$ dependent. The partial derivative of the first term is
\begin{align}\label{eqn:tmp79}
  \frac{\partial{I}}{\partial h_{lm}}
  = \sum_i\frac1{2m_i} \frac{\partial (\pi^i_\alpha h^{-1}_{\alpha\beta}h^{-1}_{\gamma\beta}\pi^i_{\gamma})}{\partial h_{lm}}
  = \sum_i\frac{\pi^i_\alpha\pi^i_\gamma}{2m_i} \Big( \frac{\partial h^{-1}_{\alpha\beta}}{\partial h_{lm}} h^{-1}_{\gamma\beta} + h^{-1}_{\alpha\beta} \frac{\partial h^{-1}_{\gamma\beta}}{\partial h_{lm}} \Big)
\end{align}
Now we need to know the partial derivative $\partial h^{-1}_{\alpha\beta}/ \partial h_{lm}$. We consider a $x$ parameterized inversible matrix $M$:
\begin{align}
  M(x)^{-1}M(x) = \textrm{Id}
\end{align}
By taking derivative w.r.t. $x$ on both sides of the identity, we have
\begin{align}
  \frac{dM^{-1}}{dx} M + M^{-1} \frac{dM}{dx} = 0
\end{align}
Therefore,
\begin{align}
  \frac{dM^{-1}}{dx} = - M^{-1} \frac{dM}{dx} M^{-1}
\end{align}
Then Eq.~\eqref{eqn:tmp79} becomes
\begin{align}\nonumber
  \frac{\partial{I}}{\partial h_{lm}}
  &= - \sum_i\frac{\pi^i_\alpha\pi^i_\gamma}{2m_i}
    \Big(
    h^{-1}_{\alpha\delta} \frac{\partial h_{\delta\epsilon}}{\partial h_{lm}} h^{-1}_{\epsilon\beta}h^{-1}_{\gamma\beta} 
    +
    h^{-1}_{\alpha\beta} h^{-1}_{\gamma\delta}\frac{\partial h_{\delta\epsilon}}{\partial h_{lm}} h^{-1}_{\epsilon\beta}
    \Big)   \\
  &= - \sum_i\frac{\pi^i_\alpha\pi^i_\gamma}{2m_i}
    \Big(
    h^{-1}_{\alpha l}  h^{-1}_{ m\beta}h^{-1}_{\gamma\beta} 
    +
    h^{-1}_{\alpha\beta} h^{-1}_{\gamma l}  h^{-1}_{ m\beta}
    \Big)   
\end{align}
The partial derivative of the potential term of \eqref{eqn:a-dph} reads
\begin{align}
  \frac{\partial U}{\partial h_{lm}} = \partial_i U \cdot  \frac{\partial \vect h\vect s_i}{\partial h_{lm}}
  = (\partial_i U)_\alpha \frac{\partial h_{\alpha\beta}}{\partial h_{lm}} s^i_{\beta}
  = (\partial_i U)_l  s^i_{m}
\end{align}
Finally, we need to know the partial derivative $\partial \det(\vect h)/\partial h_{lm}$.
We start from the identity of any matrix $M$: $\det (M) = \exp [\trace \ln (M) ]$. By taking the derivative and
noticing that $\trace( AB ) = \trace (BA)$ holds for any two matrices $A$ and $B$:
\begin{align}
  \frac{d \det(M)}{dM_{ij}} = \det ( M) \trace[ M^{-1} \frac{d M}{d M_{ij}} ] = \det(M) M^{-1}_{\alpha\beta} \frac{dM_{\beta\alpha}}{d M_{ij}}
  = \det(M) M^{-1}_{ji}
\end{align}
Therefore the partial derivative of the last term of \eqref{eqn:a-dph} reads
\begin{align}
  \frac{\partial \det(\vect h)}{\partial h_{lm}} = \det (\vect h) h^{-1}_{ml}
\end{align}
In summary, the Eq.~\eqref{eqn:a-dph} becomes
\begin{align}
  (\dot {\vect p})_{lm}
  = \sum_i\frac{\pi^i_\alpha\pi^i_\gamma}{2m_i}
  \Big(
  h^{-1}_{\alpha l}  h^{-1}_{ m\beta}h^{-1}_{\gamma\beta} 
  +
  h^{-1}_{\alpha\beta} h^{-1}_{\gamma l}  h^{-1}_{ m\beta}
  \Big)
  - (\partial_i U)_l  s^i_{m}
  - P \det (\vect h) h^{-1}_{ml}
\end{align}


The Langevin equations are
\begin{align}
  \dvs_i =\,& \frac1{m_i}\vect h^{-1}\vect h^{-\top}\mypi_i \\
  \dot{\mypi_i} =\,& - \vect h^\top \partial_i U -   \frac1{m_i} \Gamma^i\vect h^{-1}\vect h^{-\top}\mypi_i + \Sigma^i \dot W\\
  \dvh =\,& \frac1{M_h}{\vect p_h}\\ \nonumber
  (\dot{\vect p}_h)_{lm} =\,& \sum_i\frac{\pi^i_\alpha\pi^i_\gamma}{2m_i}
                           \Big(
                           h^{-1}_{\alpha l}  h^{-1}_{ m\beta}h^{-1}_{\gamma\beta} 
                           +
                           h^{-1}_{\alpha\beta} h^{-1}_{\gamma l}  h^{-1}_{ m\beta}
                           \Big)
                           - (\partial_i U)_l  s^i_{m}
                           - P \det (\vect h) h^{-1}_{ml} \\
         &- \frac1{M_h} \gamma^h_{lmpq} (p_h)_{pq}
           + \sigma^h_{lm}\dot W
\end{align}
Now we take the variable transformation:
\begin{align}
  \vect s &= \vect h^\inv \vect r\\
  \dvs &=  \vect h^\inv \dvr - \vect h^\inv\dvh\vect h^\inv \vect r\\
  \mypi & = \vect  h^\top \vect p\\
  \dot{\mypi} &= \dvh^\top\vect p + \vect h^\top\dvp
\end{align}
The Langevin equations becomes
\begin{align}
  \dvr_i &= \frac1{m_i}\vect p_i + \dvh\vect h^\inv\vect r \\\label{eqn:tmp97}
  \dvp_i & = - \partial_i U - \vect h^{-\top}\dvh^\top\vect p_i -  \frac1{m_i}\vect h^{-\top} \Gamma^i \vect h^{-1}\vect p_i + \vect h^{-\top} \Sigma^i \dot W\\
  \dvh &= \frac1{M_h}{\vect p_h}\\\label{eqn:tmp99}
  (\dot{\vect p}_h)_{lm} & = \sum_i\frac{1}{m_i}
                           p^i_l h^\inv_{m\beta} p^i_\beta
                           - (\partial_i U)_l h^\inv_{m\beta}  r^i_{\beta}
                           - P \det (\vect h) h^{-1}_{ml}
                           - \frac1{M_h} \gamma^h_{lmpq} (p_h)_{pq}
                           + \sigma^h_{lm} \dot W
\end{align}
In Eq.~\eqref{eqn:tmp97}, the matrix $\Gamma$ and vector $\Sigma$ satisfy the fluctuation-disspation theorem
\begin{align}
  \Sigma_\alpha \Sigma_\beta = 2k_BT \Gamma_{\alpha\beta}
\end{align}
We change the variables by
\begin{align}
  \hat \Sigma &= m^{-1/2} \vect h^{-\top}\Sigma\\
  \hat \Gamma &= m^\inv \vect h^{-\top} \Gamma \vect h^\inv
\end{align}
One can easily show that $\hat \Sigma$ and $\hat \Gamma$ also satisfy the  fluctuation-disspation theorem. We rewrite Eq.~\eqref{eqn:tmp97} leaving out the hats:
\begin{align}
  \dvp_i & = - \partial_i U - \vect h^{-\top}\dvh^\top\vect p_i -   \Gamma^i \vect p_i + m_i^{1/2} \Sigma^i \dot W
\end{align}
Multiplying from R.H.S. of the equation~\eqref{eqn:tmp99} by $\vect h^\top$, 
\begin{align}\label{eqn:tmp100}
  \dot{\vect p}_h \vect h^\top
  = \sum_i\frac{1}{m_i} \vect p_i\otimes\vect p_i + \vect F_i \otimes \vect r_i 
  - P \det (\vect h)\, \id
  - \frac1{M_h} (\Gamma^h \vect p_h) \vect h^\top
  + \Sigma^h \vect h^\top \dot W 
\end{align}
The pressure tensor of the system is
\begin{align}
  \vect P = \frac1{\det(\vect h)} \sum_i \Big( \frac1{m_i}\vect p_i\otimes\vect p_i + \vect F_i\otimes \vect r_i \Big )
\end{align}
And the reference pressure tensor is denoted by
\begin{align}
  \vect P_{\textrm{ref}} = P\,\id
\end{align}
Then Eq.~\eqref{eqn:tmp100} becomes
\begin{align}
  \dot{\vect p}_h 
  =
 (\vect P - \vect P_{\textrm{ref}}) \cdot (\det (\vect h) \vect h^{-\top})
  - \frac1{M_h} \Gamma^h \vect p_h
  + \Sigma^h \dot W   
\end{align}
or
\begin{align}
  \dot{\vect p}_h 
  =
 (\vect P - \vect P_{\textrm{ref}}) \cdot (\det (\vect h) \vect h^{-\top})
  -  \Gamma^h \vect p_h
  + M^{1/2}_h\Sigma^h \dot W   
\end{align}
Putting all together, the Langevin equations becomes
\begin{align}
  \dvr_i &= \frac1{m_i}\vect p_i + \dvh\vect h^\inv\vect r \\
  \dvp_i & = - \partial_i U - \vect h^{-\top}\dvh^\top\vect p_i -   \Gamma^i \vect p_i + m_i^{1/2} \Sigma^i \dot W\\
  \dvh &= \frac1{M_h}{\vect p_h}\\
  \dot{\vect p}_h 
         & =  (\vect P - \vect P_{\textrm{ref}}) \cdot (\det (\vect h) \vect h^{-\top})
           - \Gamma^h \vect p_h
           + M^{1/2}_h\Sigma^h \dot W   
\end{align}


% \bibliography{ref}{}
% \bibliographystyle{unsrt}

\end{document}

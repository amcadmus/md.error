\documentclass[aps, pre, preprint,unsortedaddress,a4paper,onecolumn]{revtex4}
% \documentclass[aps, pre, preprint,unsortedaddress,a4paper,twocolumn]{revtex4}
% \documentclass[acs, jctcce, a4paper,preprint,unsortedaddress,onecolumn]{revtex4-1}
% \documentclass[aps,pre,twocolumn,unsortedaddress]{revtex4-1}
% \documentclass[aps,jcp,groupedaddress,twocolumn,unsortedaddress]{revtex4}

\usepackage[fleqn]{amsmath}
\usepackage{amssymb}
\usepackage[dvips]{graphicx}
\usepackage{color}
\usepackage{tabularx}
\usepackage{algorithm}
\usepackage{algorithmic}

\makeatletter
\makeatother

\newcommand{\recheck}[1]{{\color{red} #1}}
\newcommand{\redc}[1]{{\color{red} #1}}
\newcommand{\bluec}[1]{{\color{red} #1}}
\newcommand{\greenc}[1]{{\color{green} #1}}
\newcommand{\vect}[1]{\textbf{\textit{#1}}}
% \newcommand{\vect}[1]{#1}
\newcommand{\dd}[1]{\textsf{#1}}
% \newcommand{\fwd}[0]{\textrm{fw}}
% \newcommand{\bwd}[0]{\textrm{bw}}
\newcommand{\fwd}[0]{+}
\newcommand{\bwd}[0]{-}
\newcommand{\period}[0]{T_{\textrm{P}}}
\newcommand{\ml}[0]{\mathcal {L}}
\newcommand{\mo}[0]{\mathcal {O}}
\newcommand{\mbp}[0]{\mathbb {P}}
\newcommand{\mc}[0]{\mathcal {C}}
\newcommand{\dt}[0]{\Delta t}
\newcommand{\id}{\mathrm{Id}}
\newcommand{\mypi}{\boldsymbol\pi}
% \newcommand{\mymu}{\boldsymbol\mu}
\newcommand{\myphi}{\Phi}
\newcommand{\mymu}{\mu}
\newcommand{\prob}{\textrm{P}}
\newcommand{\ins}{\textrm{ins}}

\newcommand{\confaa}[0]{{\alpha_{\textrm{R}}}}
\newcommand{\confab}[0]{{\alpha_{\textrm{R}}'}}
\newcommand{\confba}[0]{{\textrm{C}7_{\textrm{eq}}}}
\newcommand{\confbb}[0]{{\textrm{C}5}}
\newcommand{\confc}[0]{{\alpha_{\textrm{L}}}}



\begin{document}

\title{Structure preserving numerical scheme for isothermal isobaric Langevin dynamics}
\author{Han Wang}
\email{han.wang@fu-berlin.de}
\affiliation{Zuse Institut Berlin, Germany}
   
\begin{abstract}
\end{abstract}

\maketitle

\section{Two possible ways of defining Hamiltonian}

The isobaric simulation is implmented by rescaling the simulation region.
Here we assume the box is cubic and of volume $V$. The positions of the
atoms are denoted by $\vect r_i,\ i=1, \cdots, N$. 
The rescaling is given by
\begin{align}
  \vect s_i = V^{-\frac13} \vect r_i, \quad \vect r_i = V^{\frac13}\vect s_i
\end{align}
Therefore, the straightforward definition of the kinetic energy is given by
\begin{align}
  \mathcal K &=
  \sum_{i=1}^N \frac12 m_i \big(\frac{d}{dt} (V^{\frac13}\vect s_i) \big)^2
  +
  \frac 12 {M_V} \dot V^2 \\
  & =
  \sum_{i=1}^N \frac 12 m_i
  \big(\frac13 V^{-\frac23} \dot V \vect s_i + V^{\frac13}\dot{\vect s}_i\big)^2
  +
  \frac 12 {M_V} \dot V^2 
\end{align}
The corresponding momenta are
\begin{align}
  \mypi_i
  &=
  \frac{\partial \mathcal K}{\partial \dot{\vect s}_i} =
  m_i \big(\frac13 V^{-\frac23} \dot V \vect s_i + V^{\frac13}\dot{\vect s}_i\big) V^{\frac13}  \\
  p_V
  &=
  \frac{\partial \mathcal K}{\partial \dot{V}}
  =
  \sum_{i=1}^N m_i
  \big(\frac13 V^{-\frac23} \dot V \vect s_i + V^{\frac13}\dot{\vect s}_i\big)\,
  \frac13 V^{-\frac23}\vect s_i
  +
  {M_V}\dot V \\
  &=
  \sum_{i=1}^N 
  \frac13 V^{-1} \mypi_i\cdot\vect s_i + {M_V} \dot V
\end{align}
The Hamiltonian of the system is therefore given by
\begin{align}
  \mathcal H =
  \sum_{i=1}^N \mypi\cdot\dot{\vect s}_i + p_V \dot V - \mathcal K + \mathcal V,
\end{align}
where $\mathcal V$ is the potential energy:
\begin{align}
  \mathcal V = U(V^{\frac13}\vect s_1, \cdots, V^{\frac13}\vect s_N) + PV
\end{align}
then
\begin{align}\nonumber
  \mathcal H &=
  \sum_{i=1}^N
  m_i \big(\frac13 V^{-\frac23} \dot V \vect s_i + V^{\frac13}\dot{\vect s}_i\big) V^{\frac13}\dot{\vect s}_i
  + 
  \sum_{i=1}^N
  m_i \big(\frac13 V^{-\frac23} \dot V \vect s_i + V^{\frac13}\dot{\vect s}_i\big)
  \frac13 V^{-\frac23}\vect s_i \dot V
  - \mathcal K + \mathcal V \\
  & =
  \mathcal K + \mathcal V \\
  &=
  \sum_{i=1}^N \frac 12 m_i
  \big(\frac13 V^{-\frac23} \dot V \vect s_i + V^{\frac13}\dot{\vect s}_i\big)^2
  +
  \frac 12 {M_V} \dot V^2
  +
  U(V^{\frac13}\vect s_1, \cdots, V^{\frac13}\vect s_N) + PV \\
  &=
  \sum_{i=1}^N\frac{V^{-\frac23} \mypi_i^2}{2m_i}
  +
  \frac1{2{M_V}}\big(p_V - \sum_{i=1}^N\frac13V^{-1} \mypi_i\cdot\vect s_i\big)^2
  +
  U(V^{\frac13}\vect s_1, \cdots, V^{\frac13}\vect s_N) + PV 
\end{align}
The equation of motion:
\begin{align}
  \dot{\vect s}_i
  &=
  \frac{\partial \mathcal H}{\partial \mypi_i}
  =
  \frac{V^{-\frac23}\mypi_i}{m_i} -
  \frac1{{M_V}}\big(p_V - \sum_{i=1}^N\frac13V^{-1} \mypi_i\cdot\vect s_i\big)\frac13 V^{-1}\vect s_i \\
  \dot{\mypi}_i
  &= 
  -\frac{\partial \mathcal H}{\partial \vect s_i}
  = -\partial_i U V^{\frac13} +
  \frac1{{M_V}}\big(p_V - \sum_{i=1}^N\frac13V^{-1} \mypi_i\cdot\vect s_i\big)\frac13 V^{-1}\mypi_i \\
  \dot{V} 
  &=
  \frac{\partial \mathcal H}{\partial \vect p_V}
  =\frac1{M_V}\big(p_V - \sum_{i=1}^N\frac13V^{-1} \mypi_i\cdot\vect s_i\big)\\
  \dot{p}_V
  &=
  -\frac{\partial \mathcal H}{\partial V}
  =
  \sum_{i=1}^N \frac13V^{-\frac53} \frac{ \mypi_i^2}{m_i}
  -
  \frac1{{M_V}}\big(p_V - \sum_{i=1}^N\frac13V^{-1} \mypi_i\cdot\vect s_i\big)
  \sum_{i=1}^N\frac13V^{-2} \mypi_i\cdot\vect s_i
  -
  \sum_{i=1}^N\partial_i U \frac13V^{-\frac23}\vect s_i - P
\end{align}
These quations of motion are too complicated.  It seems there is no good way to simplify it.

We use an alternative way of defining the kinetic energy:
\begin{align}
  \mathcal K &=
  \sum_{i=1}^N \frac12 m_i V^{\frac23}\dot{\vect s}_i^2
  +
  \frac 12 {M_V} \dot V^2 
\end{align}
The momenta are given by
\begin{align}
  \mypi_i
  &=
  \frac{\partial \mathcal K}{\partial\dot{\vect s}_i}
  =m_i V^{\frac23}\dot{\vect s}_i\\
  p_V
  &=
  \frac{\partial \mathcal K}{\partial\dot{V}}
  = {M_V}\dot V
\end{align}
The the Hamiltonian is
\begin{align}
  \mathcal H
  &=
  \sum_{i=1}^N\mypi_i\cdot \dot{\vect s}_i +
  p_V \dot V - \mathcal K + \mathcal V \\
  & =
  \mathcal K + \mathcal V \\
  &=
  \sum_{i=1}^N \frac12 m_i V^{\frac23}\dot{\vect s}_i^2
  +
  \frac 12 {M_V} \dot V^2  +
  U(V^{\frac13}\vect s_1, \cdots, V^{\frac13}\vect s_N) + PV    \\
  & =
  \sum_{i=1}^N V^{-\frac23} \frac{\mypi_i^2}{2m_i}
  +
  \frac{p_V^2}{2{M_V}}
  +
  U(V^{\frac13}\vect s_1, \cdots, V^{\frac13}\vect s_N) + PV      
\end{align}
The corresponding equations of motion are:
\begin{align}
  \dot{\vect s}_i
  &=
  \frac{\partial \mathcal H}{\partial \mypi_i}
  =
  V^{-\frac23} \frac{\mypi_i}{m_i} \\
  \dot{\mypi}_i
  &=
  -\frac{\partial \mathcal H}{\partial \vect s_i}
  = -\partial_i U V^{\frac13} \\
  \dot{V} 
  &=
  \frac{\partial \mathcal H}{\partial p_V}
  = \frac{p_V}{M_V} \\
  \dot{p}_V
  &=
  -\frac{\partial \mathcal H}{\partial V}
  =
  \sum_{i=1}^N \frac13 V^{-\frac53} \frac{\mypi_i^2}{m_i}
  -
  \sum_{i=1}^N \partial_i U \frac13 V^{-\frac23} \vect s_i
  - P
\end{align}
Or we want to write it in terms of the generlized Langevin quations:
\begin{align}
  \dot{\vect s}_i
  &=
  \frac{\partial \mathcal H}{\partial \mypi_i} \\
  \dot{\mypi}_i
  &=
  -\frac{\partial \mathcal H}{\partial \vect s_i} - \gamma_i\frac{\partial H}{\partial\mypi_i} + \sigma_i \dot W\\
  \dot{V} 
  &=
  \frac{\partial \mathcal H}{\partial p_V}\\
  \dot{p}_V
  &=
  -\frac{\partial \mathcal H}{\partial V}
  -\gamma_V\frac{\partial \mathcal H}{\partial p_V} + \sigma_V \dot W
\end{align}
Equivalently:
\begin{align} \label{eq:lang-00}
  \dot{\vect s}_i
  &=
  V^{-\frac23} \frac{\mypi_i}{m_i} \\ \label{eq:lang-01}
  \dot{\mypi}_i
  &=
  -\partial_i U V^{\frac13} - \gamma_iV^{-\frac23} \frac{\mypi_i}{m_i} + \sigma_i\dot W \\\label{eq:lang-02}
  \dot{V} 
  &=
  \frac{p_V}{M_V} \\\label{eq:lang-03}
  \dot{p}_V
  &=
  \sum_{i=1}^N \frac13 V^{-\frac53} \frac{\mypi_i^2}{m_i}
  -
  \sum_{i=1}^N \partial_i U \frac13 V^{-\frac23} \vect s_i
  - P - \gamma_V  \frac{p_V}{M_V} + \sigma_V\dot W
\end{align}
Noticing that $\vect S$ and $\mypi$ are not physical variable, we do the
following variable transformations:
\begin{align}
  \vect s
  &= V^{-\frac13}\vect r \\
  \dot{\vect s}
  &=
  V^{-\frac13}\dot{\vect r} - \frac13 V^{-\frac43}\dot V\vect r\\
  \mypi
  &= V^{\frac13}\vect p\\
  \dot\mypi
  &=
  V^{\frac13}\dot{\vect p} + \frac13 V^{-\frac23}\dot V\vect p
\end{align}
Then Eq.~\eqref{eq:lang-00}--\eqref{eq:lang-03} become
\begin{align}
  \dot{\vect r}_i
  &= \frac{\vect p_i}{m_i} + \frac{\dot V}{3V} \vect r_i\\
  \dot{\vect p}_i
  & =
  -\partial_i U - \frac{\dot V}{3V}\vect p_i - \gamma_iV^{-\frac23}\frac{\vect p_i}{m_i}
  + \sigma_iV^{-\frac13}\dot W\\
  \dot{V} 
  &=
  \frac{p_V}{M_V} \\
  \dot{p}_V
  &=
  \sum_{i=1}^N \frac13 V^{-1} \frac{\vect p_i^2}{m_i}
  -
  \sum_{i=1}^N \frac13 V^{-1} \partial_i U  \vect r_i
  - P - \gamma_V  \frac{p_V}{M_V} + \sigma_V\dot W 
\end{align}
We denote
\begin{align}
  \hat \gamma_i = \gamma_i V^{-\frac23}, \quad \hat \sigma_i = \sigma_i V^{-\frac13}
\end{align}
Here the new notation satisfies the same fluctuation-disspation relation as the old one.
The instantaneous pressure is given by
\begin{align}
  P_\ins =
  \sum_{i=1}^N \frac13 V^{-1} \frac{\vect p_i^2}{m_i}
  -
  \sum_{i=1}^N \frac13 V^{-1} \partial_i U  \vect r_i
\end{align}
Therefore we end up with the Langevin equations in the transformed variables:
\begin{align}
  \dot{\vect r}_i
  &= \frac{\vect p_i}{m_i} + \frac{\dot V}{3V} \vect r_i\\
  \dot{\vect p}_i
  & =
  -\partial_i U - \frac{\dot V}{3V}\vect p_i - \hat\gamma_i\frac{\vect p_i}{m_i}
  + \hat\sigma_i\dot W\\
  \dot{V} 
  &=
  \frac{p_V}{M_V} \\
  \dot{p}_V
  &=
  P_\ins - P - \gamma_V  \frac{p_V}{M_V} + \sigma_V\dot W   
\end{align}
Please not that the variable $\vect p$ here is not exactly the same as the physical momentum $\tilde {\vect p}$:
\begin{align}\nonumber
  \tilde {\vect p}
  &= m \dot{\vect r}
  = m \frac{d}{dt}\big(V^{\frac13}\vect s\big)
  = m V^{\frac13}\dot{\vect s} + \frac13 V^{-\frac23}\dot Vm\vect s \\\nonumber
  & =
  V^{-\frac13}\mypi + \frac{\dot V}{3V}m V^{\frac13}\vect s \\
  &=
  \vect p + \frac{\dot V}{3V}m\vect r
\end{align}





% \bibliography{ref}{}
% \bibliographystyle{unsrt}

\end{document}
